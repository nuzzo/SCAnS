%%\documentclass[a4paper]{article}
%\documentclass[12pt]{article}
\documentclass[12pt]{dod-blank}

%% Language and font encodings
\usepackage[english]{babel}
\usepackage[utf8x]{inputenc}
\usepackage[T1]{fontenc}

%% Sets page size and margins
%%\usepackage[a4paper,top=3cm,bottom=2cm,left=3cm,right=3cm,marginparwidth=1.75cm]{geometry}
%\usepackage[top=1in, bottom=1in, left=1in, right=1in]{geometry}



%% Useful packages
\usepackage{amsmath}
\usepackage{graphicx}
  \graphicspath{{.}{./image/}}
  \DeclareGraphicsExtensions{.png,.jpg} 
\usepackage[colorinlistoftodos]{todonotes}
\usepackage[colorlinks=true, allcolors=blue]{hyperref}
\usepackage{tabularx}
\usepackage{multirow}
\usepackage{tabulary}
\usepackage{float}
\usepackage{wrapfig}
\usepackage[export]{adjustbox}
\usepackage{comment}
\usepackage{tabularx}
\usepackage{multirow}
\usepackage{tabulary}
\usepackage{enumitem}

\usepackage{listings}
\usepackage{color}
\usepackage{array}
\usepackage{subcaption}
\usepackage{xcolor}




\renewcommand{\textfraction}{0}
\renewcommand{\topfraction}{1.0}
\renewcommand{\bottomfraction}{1.0}

\usepackage{longtable}
%% macros
\newif\iffinal
\finaltrue
\iffinal
  
    \newcommand\baareq[1]{}
    \newcommand\baades[1]{}
 
 
\else
    \definecolor{darkgreen}{rgb}{0,0.4,0}
    \definecolor{darkcyan}{rgb}{0,0.4,0.4}
    \definecolor{darkblue}{rgb}{0,0,0.5}
    
    \newcommand\baareq[1]{{\color{darkcyan}[\textbf{Requirement:} #1]}}
    \newcommand\baades[1]{{\color{darkcyan}[\textbf{Description:} #1]}}
 
\fi




\def\naive{na\"{\i}ve}



\lstset{ 
  backgroundcolor=\color{white},   % choose the background color; you must add \usepackage{color} or \usepackage{xcolor}
  basicstyle=\footnotesize\ttfamily,            % the size of the fonts that are used for the code
  breakatwhitespace=false,         % sets if automatic breaks should only happen at whitespace
  breaklines=true,                 % sets automatic line breaking
  captionpos=b,                    % sets the caption-position to bottom
  commentstyle=\color{mygreen},    % comment style
  % deletekeywords={...},            % if you want to delete keywords from the given language
  escapeinside={\%*}{*)},          % if you want to add LaTeX within your code
  extendedchars=true,              % lets you use non-ASCII characters; for 8-bits encodings only, does not work with UTF-8
  frame=single,	                   % adds a frame around the code
  keepspaces=false,                 % keeps spaces in text, useful for keeping indentation of code (possibly needs columns=flexible)
  keywordstyle=\color{blue}\bfseries\underbar,       % keyword style
  language=Prolog,                 % the language of the code
  % morekeywords={if,and},        % if you want to add more keywords to the set
  numbers=none,                    % where to put the line-numbers; possible values are (none, left, right)
  numbersep=5pt,                   % how far the line-numbers are from the code
  numberstyle=\tiny\color{mygray}, % the style that is used for the line-numbers
  rulecolor=\color{black},         % if not set, the frame-color may be changed on line-breaks within not-black text
  showspaces=false,                % show spaces everywhere adding particular underscores; it overrides 'showstringspaces'
  showstringspaces=false,          % underline spaces within strings only
  showtabs=false,                  % show tabs within strings adding particular underscores
  stepnumber=2,                    % the step between two line-numbers. If it's 1, each line will be numbered
  stringstyle=\color{mymauve},     % string literal style
  tabsize=2,	                   % sets default tabsize to 2 spaces
  title=\lstname                   % show the filename of files included with \lstinputlisting; also try caption instead of title
}

% apply trick for additional keywords for our AC DSL
\lstset{
	emph={for, if, and, or},
    emphstyle={\color{blue}\bfseries\underbar}
}




\title{DARPA Assured Autonomy}
\author{Technical Volume- \textcolor{red}{Thirty-Eight (38) pages max}}

\begin{document}
\pagenumbering{roman}
 
\begin{center}
\large{\textbf{Volume 1: Technical and Management Proposal}}
\end{center}
\textbf{BAA Number:} DARPA-HR001117S0045 \\
\textbf{Technical Area:} TA1, TA2, and TA3 \\
\textbf{Proposal Title:} Assured Autonomy for Learning Enabled Vehicles (Safeguard) \\
\textbf{Lead Institution:} University of Southern California \\
\textbf{Type of organization: } “OTHER EDUCATIONAL” \\

\begin{tabularx}{\linewidth}{XX}

 \textbf{Technical Point of Contact} &  \textbf{Administrative Point of Contact }   \\
Dr.\ Craig A. Knoblock  & Sapphire Masterson  \\ 
USC Information Sciences Institute & USC Dept. of Contracts \& Grants \\
4676 Admiralty Way, Suite 1001 & 4676 Admiralty Way, Suite 1001 \\
Marina del Rey, CA 90292 & Marina del Rey, CA 90292 \\
Tel: 310-448-8786 &  Tel: (310) 448-9161 \\
E-mail: knoblock@isi.edu  & E-mail: sapphirm@usc.edu \\
\end{tabularx}
\\
\\
\textbf{Award instrument requested:}  Procurement Contract, Cost-Reimbursement, No Fee
\\
\\
\textbf{Total amount of the proposed effort:} \$ ...\\
Phase I: \$ ... \\
Phase II: \$ ... \\
Phase III: \$ ... \\
\\
\textbf{Place(s) of performance:} USC, Marina del Rey, CA; Los Angeles, CA;  Tempe, AZ; Palo Alto, CA \\
\textbf{Period(s) of performance:} 04/02/2018 - 03/31/2022     \\
\\
\textbf{Other team members:} \\
\begin{tabularx}{\linewidth}{XX}
Kestrel Technology  & Arizona State University \\
(small business) & (Other Educational)\\
POC: Matthew Barry & POC: Chitta Baral\\
3260 Hillview Avenue & Department of Computer Science and Engr. \\
Palo Alto, CA 94304 & Ira A. Fulton School of Engineering \\ 
phone: (832)205-4876 & Arizona State University\\ 
mrbarry@kestreltechnology.com & Brickyard Suite 572, 699 S. Mill Avenue \\
& Tempe, AZ 85281-8809, U.S.A.\\
& email: chitta@asu.edu\\
\end{tabularx}
\\
\textbf{Proposal validity period: } 180 days\\
\\
\textbf{Data Universal Numbering System (DUNS) number: } 072933393\\
\textbf{Taxpayer identification number:} 95-1642394\\
\textbf{Commercial and Government Entity (CAGE) code:} 1B729 Marina del Rey, CA\\
\textbf{Proposer’s reference number (if any):}  4409-0\\

\newpage
\section{Table of Contents}
\tableofcontents

\newpage
\pagenumbering{arabic}
\section{Executive Summary}
As we rapidly move into a world where machine learning plays a central role in realizing autonomous systems, it is becoming increasingly important to develop techniques that assure that these systems will operate safely and perform as expected. Current approaches are limited to providing assurance for systems with limited or no  learning capabilities. In this context, DARPA's Assured Autonomy BAA seeks to \emph{develop rigorous design and analysis technologies for continual assurance of learning-enabled autonomous systems}. USC in collaboration with Kestrel Technology and ASU is pleased to submit a comprehensive TA1, TA2, and TA3 proposal entitled \emph{``Assured Autonomy for Learning Enabled Vehicles (Safeguard).''} We plan to provide an end-to-end solution to support autonomous systems with learning-enabled components, ranging from design technologies for assurance, to assurance monitoring and control techniques, to representation and online evaluation of assurance cases. We have assembled a strong team of experts that cover the range of technologies that are required to create such an end-to-end system. If successful, the project will provide the technologies for building the next-generation of learning-enabled autonomous systems.  The entire project will take four years and cost \textcolor{red}{\$??}, with an initial version completed at the end of Phase I and successive versions with additional capabilities and improved scalability at the end of Phase II and Phase III.  

In the remainder of this section, we first introduce an  unmanned surface vehicle scenario that will be used throughout the proposal to describe the approach.  Next, we describe our approach to design, monitoring, and dynamic assurance. Finally, we introduce the team involved in the project. 

\textbf{Motivating Scenario.} Consider an autonomous unmanned surface vehicle (USV) guarding a valuable asset in the ocean when an unknown vehicle  approaches the security perimeter, under challenging weather conditions. In this scenario, the USV is required to approach the intruding vehicle, issue a warning signal, and escort it to a safe distance from the controlled area. However, as the USV has no a priori knowledge of its external environment behaviors (e.g., water depth, waves, wind, current, visibility), pre-computing a feasible trajectory, let alone optimal, becomes a non-trivial problem. For trajectory planning, the USV must continuously perform the following tasks:
\begin{itemize}[itemsep=0pt,leftmargin=*]
 \item Sense the current state of the surrounding environment (e.g., water depth, waves, wind, current, visibility) and estimate its own maneuverability constraints (e.g., braking distance, available acceleration, maximum velocity, turning radius, turning rate, safety distance) based on the state of the environment;      
\item Sense the static obstacles in the sensor range and generate a traversability map;
\item Sense the moving obstacles and classify them;   
\item Predict future trajectories of moving obstacles; 
\item Determine if any of the COLREGS \cite{commandant1999international} rules will be in effect with respect to one or more of the nearby vessels and identify the vessels with the right of way.    
\end{itemize}
The above information will be used by the trajectory planner to compute an initial trajectory, which will be continuously refined as the USV gathers additional information.
% It is not possible for the USV to be tested in every possible environment. 
The USV will use learning enabled components to take  decisions as it encounters new situations, such as  
\begin{itemize}[itemsep=0pt,leftmargin=*]
\item Classifiers to identify moving obstacles based on physical appearance and motion signatures,
\item Algorithms to estimate the sensor capabilities in adverse weather conditions,   
\item Algorithms to accurately estimate uncertainty in the environment, 
\item Classifiers to generate traversability maps,
\item Prediction of external vessel behaviors based on motion histories, 
\item Reinforcement learning  to ensure COLREGS compliance of maneuvers,  
\item Algorithms to learning pursuit behaviors.  
\end{itemize}
Learning enabled components will interact with each other in complex ways, where a misclassification error in one component may eventually compromise the entire mission.   
% We will need to make sure that each learning enabled components has a run-time monitor that will ensure that the assumptions made by the learning-enabled component remain valid and prevent erroneous learning. 
% For example, if the vehicle is exhibiting significant error in trajectory tracking, then simply downgrading the trajectory tracking error value may not be a good option.  The failure of prediction of trajectory tracking error might be due to the presence of a significant wake caused by a nearby vessel. The presence of the nearby vessel can be used to explain the degradation in trajectory tracking performance. As the vessel moves away, we can expect the trajectory tracking performance to return to the predicted level.  
While exhaustive validation of learning-enabled cyber-physical systems (LE-CPSs) is a prohibitive task~\cite{Kalra16},
their complexity, heterogeneity, and highly dynamic nature
make it challenging to even leverage existing model-based development techniques to effectively assess system correctness 
% dependability, 
at design time or enforce it at runtime.

\textbf{Design for Assurance.} Safeguard uses a platform-based design approach~\cite{Nuzzo15b} to organize the design process for a LE-CPS and to build assurance cases. Composite models are developed at several levels of abstraction,
from top-level system requirements and safety constraints down to the
implementation level.  Intermediate levels add detail to the levels
above.  The different levels are connected by refinement mappings that
allow properties established at one level to be preserved at the next
level (see Figures~\ref{fig:methodology} and~\ref{fig:assurance}).

Contracts are used to formally specify components and composite models
in terms of (1) Assumptions -- the assumed behaviors of the
environment and the behaviors of other components, and (2) Guarantees
-- the behavior properties that a model guarantees if it operates in a
context that satisfies its assumptions.  A calculus of contracts
allows horizontal composition of contracts to generate contracts for
composite models.  Vertical contracts are used to specify the mapping
or refinement relation between models at different levels of
abstraction.  The system design process starts with a high-level
contract that expresses overall system assumptions and requirements.
Subsequent levels express models with increasing detail until the
lowest level expresses the system in terms of hardware components and
their software controllers.

The assurance case for a CPS arises from the horizontal and vertical
structure of the design in several ways.  The components used within a
particular level are either (1) synthesized using
correct-by-construction design tools together with proofs, (2) derived
statically or dynamically using safety-aware machine-learning
techniques, (3) written manually and verified by analysis tools, or
(4) written manually and validated by extensive testing.  The
assurance case for the whole reflects its compositional structure.  We
anticipate that well-specified contracts together with the calculus of
contracts will eliminate well-known problems with unexpected emergent
behaviors in CPS systems.

The assurance case for the lowest-layer design arises from both the
intra-level assurance and from properties and their proofs that are
preserved under the refinement mapping from the top-level
requirements.  The refinement mappings between model layers will be
constructed using a variety of techniques.  A contract at an abstract
level can be mapped to a component or refined contract by (1)
retrieval of pre-verified components from a platform library, (2)
synthesis using correct-by-construction design and optimization tools,
or (3) manual coding to satisfy a contract.  The mapping of a
composite model will be composed from the mappings of its constituent
components or contracts.  When a composite model cannot be mapped
compositionally to the next level, it will be generated using
correct-by-construction design and optimization tools.

\textbf{Assurance Monitoring and Control.}
We provide an integrated framework for safety-aware learning, assurance monitoring and control, detecting distribution shifts. Three major components offer an efficient TA2 architecture as well as interfaces with TA1 and TA3, that is, (a) safety-aware learning and planning, (b) assurance monitors for guarding architectural and safety constraints; and (c) distribution shift detection.

We will develop a new learning-enabled online decision-making framework that allows opportunistically composing a sequence of actions (maneuvers) to reduce uncertainty in the system capability model without suspending the progress toward the mission goals or compromising safety. Each candidate action is evaluated based on three criteria: (1) the risk of violating a safety constraint using the current uncertainties in the parameter estimates; (2) its relevance to the mission goals; (3)  its expected information gain, i.e., reduction in uncertainty, with respect to the parameter estimates. These evaluations are combined to produce a cumulative mission utility value for each action that drives our learning-enabled decision-making framework. The problem of generating and evaluating sequences of actions can be posed in several way. For example, it can be solved using a branch-and-bound search method like Anytime A*, or formulated with the finite-horizon Markov Decision Process (MDP) framework. We will develop new scalable search strategies to solve this problem efficiently, by potentially evaluating a recent method developed at USC, called FastMap, that can significantly improve the execution time. 

We will develop monitors for architectural and safety constraints. 
% While these constraints can be checked over and over again as sensor information flow in, this naive strategy accounts for a lot of computational overhead. 
To achieve scalability and decrease the overhead, we propose the application of a technique that we currently use in DARPA's RSPACE program, which leverages a physical model of the vehicles dynamics and its interactions with the environment to efficiently determine the readout frequency. We propose two  extensions of this basic idea. First, we will use the theory of Variable Elimination to prioritize which variables to monitor, e.g., controllable, versus uncontrollable, adversarially controlled, or unobservable variables. Second, we invoke the dynamic assessment of assurance cases only when needed. This  decreases the number of times dynamic assessment of assurance cases is initiated as well as the communication bandwidth between the TA2 and TA3 components.

Finally, we will identify a distribution shift by combining statistical and machine learning techniques to differentiate between environmental and sensor changes. We will exploit a categorization of the shifts based on their cause and duration as well as extend our earlier work on detecting and mitigating sensor failures for all types of monitored variables.  

\textbf{Dynamic Assurance:} The Safeguard {\em design for assurance\/} activity takes a systems-theoretic stance toward safety.  Consequently, it presumes that safety is an emergent property of the system, and that hazards can present themselves through unintended interactions and performance violations in addition to causal events such as component failures.  Our design approach includes consideration of intent as well as hazard analysis and mitigation.  The artifacts from these activities populate contracts and assumptions for the dynamic assurance case.  
We thus build safety into the product by working at a systems-level viewpoint, using lexicon and design patterns familiar to both hardware and software engineers; safety is an emergent property of the system, not an afterthought.  
As system behavior evolves during runtime owing to learning, threats, degradation, or some other factor, the dynamic assurance case identifies whether the safety constraints continue to be satisfied.  If not, it provides notifications or issues recovery instructions directly from a lookup table.

Our implementation of the dynamic assurance case employs a declarative knowledge base inference engine and a domain-specific language tailored to our approach.  We have used them successfully for assurance case tool sets and arguments, and will extend them to reason about uncertainty and learning.  Our approach to achieve scalability is to specialize solvers toward modularity and to take advantage of domain knowledge.  Specifically, we will develop answer set programming techniques for context-dependent learning for reasoning about the learning-enabled components as well as learning assurance rules.  We will develop new formalisms for uncertainty to include causality, using weights for computing probabilities, and probabilistic non-monotonicity.  To achieve scaling objectives we will implement specializations using modularity, weighted CSPs, and message passing. 

% The system safety constraints revealed from that design become the key elements of our dynamic assurance case.  Our verification tools ensure the constraints are relevant, identifiable, and their implementation and effect observable.  

\textbf{Team.} We have assembled a team that is exceptionally well-qualified to build the proposed Safeguard system.  The team will be led by Dr.\ Craig Knoblock, the Principal Investigator for the effort, who currently leads the Intelligent Systems Division at the Information Sciences Institute.  He has led many large DARPA and IARPA projects over the years and has a strong track record in conducting leading edge research and then transitioning the technology to commercial use.  He will be supported by Dr.\ Michael Orosz as the Project Manager, who also has  experience in managing large research projects and on autonomous systems.  The TA1 team will be led by Dr.\ Pierluigi Nuzzo, who is an expert in embedded system design methodologies and the  application of formal methods to cyber-physical systems.  The TA1 team also includes Dr.\ Doug Smith, who has spent many years working on scalable correct-by-construction techniques and Dr.\ Henny Sipma, who has significant experience in applying program verification methods to real-world problems.  The TA1 team also includes Ryan Goodfellow, who has done a large amount of work on simulation-based testing.  The TA2 team will be led by Dr.\ Knoblock who has worked on topics related to both monitoring and detecting distribution changes.  He will be supported by Dr.\ Satyandra Gupta, who is an expert on autonomous surface vehicles as well as on safety-aware learning. He will also be supported by Drs.\ Anoop Kumar and Satish Thittamaranahalli, who have also previously worked on efficient methods for execution monitoring.  The TA3 team will be lead by Dr.\ Matthew Barry, who has experience in creating the technologies for assurance cases.  He will be supported by Dr.\ Chitta Baral, who is an expert on ASP solvers and by Dr.\ Thittamaranahalli who is an expert on SAT solvers, both of which will be applied to provide scalable assurance case reasoning.  Finally, Dr.\ Petros Ioannou, who is an expert on control systems for autonomous vehicles will provide an autonomous vehicle platform, which will form the focus of our work until the TA4 teams provide additional vehicle platforms for development.  

\newpage
\section{Innovative Claims and Deliverables}

In this project we will develop and build an end-to-end system for assured autonomy.  This section describes the key innovations by technical area and then the overall deliverables of the project.

\paragraph{Design for Assurance}

\begin{itemize}[itemsep=0pt,leftmargin=*]
\item We address the LE-CPS design challenges via a holistic approach that can contextually generate design artifacts and assurance cases. We develop a compositional, contract-based modeling framework, methods, and tools to support the design process from system-level requirement capture,  formalization, and analysis, to the generation, testing, and continual monitoring of software and hardware artifacts in feedback loop with a physical process.

\item We develop compositional abstractions and interfaces (vertical contracts) that can  bridge heterogeneous formalisms and heterogeneous decomposition architectures to make system analysis and synthesis tractable, consistently combine different verification and synthesis methods at design time, and provide seamless support for dynamic assurance at run time. %We aim to quantitatively capture the confidence in the satisfaction of requirements under uncertain or unknown conditions, and resilience properties of  systems at different abstraction levels, to enable trade-off evaluation between resilience, performance, and cost.

\item We develop a unifying framework and efficient algorithms to reason about the combination of discrete and continuous dynamics and constraints in the presence of uncertainties in LE-CPS using a satisfiability modulo convex approach~\cite{Shoukry2017} for contract-based system verification and scalable trajectory planning.  

\item We provide an environment for high-fidelity CPS testing, in which production-ready software, e.g.,  safety-critical learning and control, may be deployed and tested 
% by extending the Cypress testbed environment \cite{Goodfellow2015Cypress:Systems} 
with time dilation facilities, so that it synchronizes with a physical simulation that is not necessarily running in real time, while still having the perception of real time.

\item We 
% These facilities allow a cyber system to be  
propose an approach for unanticipated behavior space identification and test coverage maximization which leverages results from the theory of differential algebraic equation (DAE)~\cite{Berger2013ControllabilitySurvey,Ilchmann2005ATheory,BergerOnSystems,Lamour2013} 
to prune the behavior search space and identify smaller regions of interest for efficient simulation-based testing. 
% We then compute the intersection of these two behavior spaces and restrict our simulation based testing search space to this subspace.
\end{itemize}

\paragraph{Assurance Monitoring and Control}

\begin{itemize}[itemsep=0pt,leftmargin=*]
\item 
%We integrate safety-aware learning into the overall decision making problem. The goal is to maximize mission utility without violating the safety constraints. 
Our safety-aware learning framework enables the system to opportunistically select and execute actions to assist the learning-enabled component in reducing model uncertainty without compromising safety or deviating from the mission goals. The value of uncertainty reduction is explicitly incorporated in the optimization process for selecting the best action.  
\item For safety-aware learning, we propose the idea of preprocessing the search space of the problem domain before queries and observations come in. With such a linear-time preprocessing phase, the performance of search and optimization algorithms can be significantly boosted. For example, in regular A* search, the intensional or extensional search space can be preprocessed in near-linear time to yield an embedding of each state as a point in Euclidean space~\cite{cujakk}. Then, when the query comes in, A* search can make use of these Euclidean distances as heuristic distances between two states to yield order-of-magnitude speedups. 
%In Anytime A* for safety-aware learning and planning, this leads to a significantly better quality of actions chosen within a time limit, and in the MDP framework, the same ideas can be used to improve the convergence of Bellman updates for safety-aware Reinforcement Learning.
\item As massive amounts of sensor information flow in, it is imperative for us to efficiently process this information for monitoring architectural and safety constraints. Building on our past work on similar tasks, we propose novel technologies for efficiently monitoring constraints. These algorithms can yield an exponential reduction in the amount of sensor data that needs to be processed. Doing this also reduces the message complexities between the various modules. %We also propose to use the theory of Variable Elimination (VE) to monitor constraints with uncontrollable, adversarially controlled, and/or unobservable variables. VE yields a substrate constraint to monitor that characterizes a dominant strategy of the controllable variables over the uncontrollable, adversarially controlled, and/or unobservable variables.
\item We will develop techniques to identify  distributional shifts and determine the underlying cause (e.g., change in environment, sensor failure,   etc.), as well as strategies for handling the various distributional shifts.   Notably, we propose to build on our past work and use compact representations to exploit historical data to identify distributional shifts.
\end{itemize}

\paragraph{Dynamic Assurance}

\begin{itemize}[itemsep=0pt,leftmargin=*]

\item We demonstrate the integration of dynamic assurance for safety-critical learning-enabled dynamic systems in which evolutionary behaviors are expected and tolerated as a property of the functionality.   The impact will be consequential contributions safety-critical dynamic systems in which evolutionary behaviors are expected and tolerated as portion of the functionality.   
\item We implement dynamic assurance by combining features of system safety, formal methods, logic programming, uncertain reasoning, and domain-specific languages.  We populate assurance case arguments at several levels of modeling and implementation abstraction, using the analysis results to produce design-time evidence supporting assurance claims.  
%We provide automated reasoning about the assurance case itself to produce verification, consistency, and completeness results for the argument.  Dynamic assurance results then yield trusted explanations of whether safety constraints and assumptions and other contracts still hold during the collection of runtime evidence from monitors. 
\item We develop and demonstrate ASP formalisms crucial to applications in dynamic assurance. We demonstrate the suitability of the technology especially for assurance case arguments owing to the improved legibility, consistency and completeness checks, handling of uncertain and default reasoning, and scalability.  
%We will produce modularized solvers for enhanced performance based on recent algorithmic developments in exploiting structure, kernelization, and message passing. We provide a formalism to enable learning of assurance rules. 
We provide a novel approach to handling uncertainty that provides the ability to do causal and counter-factual reasoning as well as probabilistic non-monotonicity.  Overcoming limitations of traditional inductive logic techniques, we develop a novel iterative and incremental approach based on context dependent learning. 
\end{itemize}

\paragraph{Deliverables}
During the course of this project, we will build and deliver a fully-operational system that covers all three of the technical areas.  The detailed capabilities of this system are described in the individual technical sections.  The resulting system will be available as open source under a permissive license, which will allow other organizations to use the work, extend it in new directions, and even commercialize the software.  Kestrel Technology has significant experience in this space and has built and applied these types of technologies to a variety of real world tasks.  Kestrel is ideally suited to pursue commercial uses of this technology and the permissive license will facilitate exploring these opportunities since there will be no need to negotiate intellectual property rights.  

\newpage
\section{Technical Plan}



\subsection{Dynamic Assurance}

\subsubsection{Current Approaches and Issues} 
% Suggested relocation color code: 
%\textcolor{orange}{Move to Overview}
%\textcolor{blue}{Move to TA1}
%\textcolor{cyan}{Move to TA2}

%\begin{itemize}
%\item Briefly describe current approaches
%\item Discuss challenges/issues with these approaches
%\item What makes our approach innovative with respect to the state-of-the-art
%\end{itemize}

% \subsubsection{Current Approaches and Issues}

A recent survey of safety engineering compliance practices relates to our dynamic assurance case  challenge (\cite{RESSJ2013}).  The survey found that practitioners frequently relied upon {\em process-based evidence}, such as verification and validation plans and on safety management plans. When the practitioners provided {\em product-based\/} evidence they provided the requirements specification and test results. The study authors mention their intrigue that evidence types concerning risks and hazards are not among the most frequently reported. 
% Practitioners most often use checklists and expert judgment (with rationale) for assessing evidence.  
Their highest challenge is determining the confidence in evidence supporting a claim about system safety. 
% A critical challenge is gathering the design and development artifacts along with the decision process involved to collect them.  
Formal verification results were used by fewer than 30\% of the study respondents.  Completeness of the evidence is checked {\em manually\/} and through a manual predefined process, according to more than half of the respondents. 

Regarding the structure of an assurance case, it has been popular for many years to refer to legal reasoning and graphical notations (\cite{kelly04}).  
% For small arguments, these graphical notations illustrate how to go about structuring and supporting an argument for the trustworthiness of the system.  
The tree-structured {\em goal structuring notation\/} (GSN) is popular, and its cousin dubbed the {\em claim-argument-evidence\/} is an alternate representation.  
% Again for small arguments, one could question whether these are an improvement over free-form text.  
For larger arguments, the tree structures quickly become undesirable, fine for computer internal representation but clumsy for human-computer interfaces or printed documents.  Moreover, there is no specification for the automated analysis of these trees, leaving the user to impose writing requirements to facilitate analysis (\cite{rushby10}).  Somewhere in-between graphical notations and free-form text is semi-structured text (\cite{CMK08}), a DSL for writing case arguments while lending support for fancy editors and automated analysis.

Interesting assurance case structures that lend themselves to analysis have been developed.  
% We describe here the Resolute language, the Evidential Tool Bus, and the $L$ domain specific language (DSL).  
The Resolute language consists of both a logic and a computational sublanguage (\cite{Resolute}). The logic of Resolute is an intuitionistic logic similar to pure Prolog, but augmented with explicit quantification. 
% The logic is parameterized by the computational sublanguage, and requires only that the sub-language is deterministic. 
% This allows the computational sublanguage to be customized to any domain, such as AADL, and to be expanded and refined, without worrying about the logical consequences.   
The Evidential Tool Bus (ETB) (\cite{ETB}) is similar in syntax and semantics to Resolute. It provides a Datalog-style logic and is designed to combine evidence from a variety of sources. However, the focus of the ETB is on distribution and on provenance.  
% It logs the sequence of tool invocations that were performed to solve a query. 
% It uses timestamps to determine which analyses are out of date with respect to the current development artifacts and to only re-run those analyses that are not synchronized with the current development artifacts. 
% ETB is meant to be tool and model agnostic, and therefore has no particular knowledge of the system architecture.  
The DSL $L$ takes a semi-structured text approach (\cite{CertWareABSA}).  It provides an English-like language having a reasonable syntax and semantics for authors to write cases in an argument structure, but also provides analytic support for the document.  The $L$ tool set provides an answer set solver to reason about the case argument and return results to the author. 


% \subsubsection{Issues with Current Approaches}

The safety practitioner survey revealed that simple textual templates are the most frequently found technique for evidence structuring. Study respondents reported rarely using argumentation-based graphical notation such as goal-structuring notation (GSN) or process models such as the OMG's software and system process engineering metamodel (SPEM).  Regarding argumentation-based graphical notations, the study authors note ``[t]he results suggest that a lot of research effort has been spent on a technique that has seen little industrial adoption thus far''.  We speculate, based on our own experience building tool sets for such graphical models, that the problem of quickly becoming unwieldy when modeling real-world products undermines the appeal of these techniques. 

The notation problem can be addressed by domain-specific languages that enable analysis support.  We introduced {\em answer set programming\/} (ASP) technology to provide this support \cite{CertWareABSA}, illustrating analysis of arguments, but the implementation lacks strong formalisms for handling learning, reasoning about uncertainty, and performance scalability.  Although there are some existing algorithms for learning assurance rules in ASP \cite{ray2009nonmonotonic,athakravi2013learning,law2014inductive,athakravi2015inductive,iled,kazmi2017improving}, they do not work well with large data sets. The best performing among them is the system XHAIL.  For handling uncertainty popular formalisms such as Markov Logic Networks or its recently proposed ASP variant $LP^{MLN}$ do not provide for non-monotonic specification of randomness.  An ASP-specific variant P-log does have the required construct, but lacks in its ability to specify arbitrary weights.  

The LE components introduce unique issues for safety assurance~\cite{SafetyANN}. For example, neural networks lend themselves only to black box analysis but are wanted under verification and validation regimes that require white box analysis.  Hazard analysis cannot be performed in the usual manner owing to the way the network algorithms work.  Implicit, process-based safety assurance is insufficient for neural networks owing to a lack of concern for the functional behaviors of the software.  Explicit, product-based arguments such as proposed herein are more appropriate to consider the functional behaviors. 
% To support the top-level assurance claim for an ANN component we must provide the context for the claim and its support.  The context defines characteristics of the model, appropriate application for the design, and a definition of an {\em acceptably safe\/} threshold. 




\subsubsection{Approach}

\paragraph{Domain Specific Language}

We refer to assurance cases having the conventional structure of a legal argument justifying trustworthiness for the system, in a {\em claim-argument-evidence motif}.  We populate the assurance case differently from most in that we do not use the conventional tree structures.  
% Instead we use declarative observations gathered from the physical system and its requirements,  assumptions, and constraints.  
Our declarative approach employs a DSL as semi-structured text that lends itself to automated analysis.  
We use the design and verification content to populate the assurance case argument for safety property claims and evidence at build-time, coupled with run-time monitors produced alongside the evidence to be checked during operation.  Both the build-time evidence and the production evidence in the learning-enabled environment integrate with the dynamic assurance case to maintain the argument for the trustworthiness of the system.  

We will enhance an existing assurance case DSL called $L$. The DSL works in conjunction with an answer set program (ASP) solver to check assurance case arguments.  We choose ASP technology here over other solvers such as SAT or SMT because, after grounding or translation, it provides better access modeling and specialization opportunities for domain-specific scalability.  When the dynamic argument solution shows that trustworthiness can no longer be established, the the answer set solution will show the conflict by way of admitting multiple models.   We bring the language, the editors, and solvers forward as our starting point for our dynamic assurance case (DAC) implementation.

Leveraging our systems-theoretic approach to design and implementation, we reason about the essential elements of {\em intent\/} and {\em state\/} in the DAC.  Our DAC focuses on reasoning about the states of the system because these states are essential for understanding and managing the constraints imposed for LE-CPS safety.  Moreover, through the DAC we adorn the states with associated design contracts, safety constraints, assumptions, intentions, and interactions among the states.  
% Each state is a template consisting of a state variable history, an estimator, a controller, and if interacting with the world a hardware adapter. While normally such a template facilitates methodical design activities, we adopt it for DAC elements as well.  
For human interaction and debugging, when we provide system design intent in terms of state we can use this information from the DAC to support diagnostic explanations.  The current $L$ language resembles a blend of English and Prolog, and we provide editor support to emphasize ease of use.  


% As an example consider a sample assurance case fragment (Figure~\ref{fig:ta3:tc}) from an $L$ argument for design tool chain verification with the goal of claiming certification credit~\cite{FACTS16}.  The case author proceeds from a high-level risk specification ({\tt invalidity\-Risk}) to refine the many instances (e.g. {\tt mitigated} and {\tt verification\-Risk}) involved in the argument.  
% \begin{figure}[tbp]
% \lstset{language=Prolog}
% \begin{lstlisting}
% invalidityRisk(certification(property P, software S)) if
% 	reliesOn(certification(P,S), evidence E) and 
%     produces(toolChain TC, E, P, S) and
%     verificationRisk(risk R, P, S, TC) and
%     not mitigated(R, P, S, TC).
    
% verificationRisk(unsound(modelChecker), property P, software S, toolChain T) if
% 	modelOf(S, model M) and
%     executes(modelChecker MC, input(P, M), output(verified), T) and
%     produces(T, evidence E, P, S) and
%     includes(E, verified(P, M, MC)).
    
% mitigated(unsound(modelChecker), property P, software S, toolChain T) if
% 	modelOf(S, model M) and
%     executes(modelChecker MC1, input(P, M), output(verified), T) and
%     executes(modelChecker MC2, input(P, M), output(verified), T) and
%     MC1 != MC2.

% type modelChecker = {modelChecker1, modelChecker2}.
% type risk = {unsound(modelChecker)}. 
% ...
% \end{lstlisting}
% \caption[Fragment listing of $L$ tool chain argument]{A fragment assurance case listing written in the $L$ domain-specific language. }
% \label{fig:ta3:tc}
% \end{figure}

\begin{wrapfigure}{R}{0.6\textwidth}
\vspace{-30pt}
% \begin{figure}[tbhp]
\begin{center}
% \includegraphics[width=0.8\textwidth]{system_model}
\includegraphics[width=0.6\textwidth,trim={0in 0.2in 0in 0in},clip]{./TA3/dac-sources}
\end{center}
\caption[Assurance case sources]{The system design and runtime activities provide different kinds of content, at different levels of abstraction or refinement, to the dynamic assurance case. }
% \caption[Assurance case sources]{System safety models at different levels of abstraction.  For the state model, instantiations of the pattern for each system state help to specify the system models and to communicate system safety data to the assurance case.  For the system model, control loops around a controlled process help capture the design constraints for the DAC.  } 
\label{fig:statemodel}
% \end{figure}
\vspace{-25pt}
\end{wrapfigure}

\paragraph{Assurance Solver Implementations}

We will improve one or more reasoning engines for our DAC DSL to improve their suitability for dynamic assurance.  There exists several excellent inference engines for basic ASP. 
These include Clingo~\cite{clingo} and Potassco~\cite{potassco}, smodels~\cite{smodels}, and DLV for basic ASP and some systems for languages such as P-log and $LP^{MLN}$ that augment ASP with probabilities and weights. We anticipate building specialized solvers, rather than general-purpose solvers, to support our goals for the program. 

\noindent\underline{Supporting Learning}

During Phase I, we will modify $L$ to accommodate learning of assurance rules.  To have a common formal representation of various modules and components of the learning enabled autonomous systems, we aim to develop learning enabled components that can be characterized using the common representation. 
% For a seamless integration we aim to develop tools and algorithms that can learn rules in ASP and $L$ from data. 

The general sub-area of learning logic rules from data is called inductive logic programming (ILP). 
In statistical machine learning, it is common to learn a function from a series of $\langle x, y \rangle$ pairs where $x$ denotes the input and $y$ denotes the desired output. On the other hand, ILP deals with learning a logic program $H$ given some background knowledge and a dataset containing positive and negative examples. More formally, Given a set of positive examples $E^{+}$, negative examples $E^{−}$ and some background knowledge $B$, an ILP algorithm \cite{muggleton1991inductive} finds a hypothesis $H$ such that, $B \cup H  \models E^{+}$, $B \cup H \not \models E^{−}$. The possible hypothesis space is often restricted with a language bias that is specified by a series of mode declarations $M$.

Because of the mismatch between statistical machine learning and ILP definitions, one needs to convert all the $\langle x,y \rangle$ to create a global $E^{+}$ and $E^{−}$ and extra care needs to be taken so that different $\langle x,y \rangle$   pairs do not interfere with each other.  This approach fails to scale up. That is because, when we have a large  number of $\langle x,y \rangle$  pairs, the ILP solvers will then have to deal with large ASP programs,  We propose to develop a novel iterative and incremental approach where instead of making the above conversion, the learning of rules is done by iteratively going over each example $\langle x,y \rangle$ pair.   Our approach is based on {\em context dependent learning}. By changing the inductive logic programming paradigm to context dependent learning, we will be avoiding dealing with large examples that is made of combining the set of examples. This allows us to pursue an iterative and incremental approach making our approach scalable.  We expect an order of magnitude or more improvement. 
% In this formulation, each example $E_i$ directly corresponds to an $\langle x,y \rangle$ pair and it takes into account that there are several distinct examples in a dataset. 
%We aim to develop an algorithm that will find the solutions incrementally. First it will compute the solutions of $ILP^{ctx}(\langle B,M,{E_1}\rangle)$, then it will use those to find the solutions $ILP^{ctx}(\langle B,M,{E_1, E_2} \rangle)$ and so on until it finds a solution for the original problem, $ILP^{ctx}(\langle B,M,{E_1,...E_n}\rangle)$. Thus, our algorithm will not compute the solutions of $ILP^{ctx}(\langle B,M,{E_1, E_2,..E_i}\rangle)$ in a single step. Rather it will compute several lower bound-upper bound pairs on the search space and employ a state space search for each lower bound-upper bound pair until all the minimal solutions in that space are found. 


% , defined as follows:
%{\bf Definition}:	A \textit{Context Dependent Learning} task $ILP^{ctx}$ is a tuple $\langle B,M,D \rangle$, where $B$ is an Answer Set Program, called the background knowledge, $M$ defines the set of rules allowed in hypotheses (the hypothesis space) and $D$ is the dataset containing a series of context dependent examples $E_1,E_2,..., E_n$. Here each $E_i$ is a tuple $\langle O_i,E_{i}^{+}, E_{i}^{-} \rangle$ where,  $O_i$ is a logic program, called \textit{observation} , $E^{+}$ is a set of positive ground literals and $E^{-}$ is a set of negative ground literals. A hypothesis $H$ is an inductive solution of $T$ (written as $H \in ILP^{ctx}(B,M,D)$) \textit{iff}
%	$H\cup B \cup O_i \vdash E_{i}^{+}, ~\forall i=1...n$,  and   
%	$H\cup B \cup O_i \not \vdash E_{i}^{-}, ~\forall i=1...n$. 

Another learning aspect we will address is allowing previously undefined predicates in $H$. Currently, in all ILP systems and algorithms the mode definitions of rules in $H$ do not allow previously undefined predicates. 
% However, such predicates are needed to learn rules whose conditional part may contain universal quantifications. 
The universal quantification plays an important role in the specification of temporal operators ``{\em always\/}'' and ``{\em until\/}''.

\noindent\underline{Supporting Uncertainty}

During Phase II, we will add support for uncertainty. 
%In recent years several extensions of ASP have been proposed to account for uncertainty.  One approach, as used in P-log, enhances ASP with some explicit probability assignments and additional implicitly derived probability numbers. In another approach, as used in LP\^MLN (and even in the much earlier introduced ASP with weak constraints), weight assignments are given explicitly and probability numbers are computed on top of them.  
We propose to develop a new formalism that incorporates reasoning features crucial to assurance with respect to autonomous systems, such as causality, counter-factual reasoning, use of weights for computing probabilities, 
% instead of direct probability numbers, which are hard to estimate for assurance rules) from which probabilities are computed, 
and probabilistic non-monotonicity -- the ability to have new possible models as a result of new information. 
Probabilistic non-monotonicity is useful for non-\naive\ conditioning, as \naive\ conditioning may fail with new sensors that ask different kind of questions and as a result need a new set of possible worlds \cite{halpern2003, baral2007}. 
% (Such a situation is described in \cite{halpern2003} and addressed in the context of ASP by us in \cite{baral2007}.) 

%To allow non-\naive\ conditioning we need rules of the form:
%\begin{equation*}
%{\tt random}(o(Y) : \{X : p(X)\}) :- a_1, \ldots , a_n, \ {\bf not} \ b_1, \ldots, \ {\bf not} \ b_m. 
%\end{equation*}
%The above rules state that $o(Y)$ is a random variable where $Y$ can take the value from the set \{X : p(X)\} when $B$ is true. Such a construct allows non-monotonic specification of when $o(Y)$ is a random variable and what its extent is. Such non-monotonic specification of randomness is not possible in the popular uncertainty formalisms such as Markov Logic Networks or its recently proposed ASP variant $LP^{MLN}$. While P-log has such a construct, it lacks in its ability to specify arbitrary weights. 

\noindent\underline{Supporting Scaling}

During Phase III, we will focus on ASP solver speed improvements.  There exist excellent inference engines for basic ASP, but none of them has emphasized speed.  We will develop a more efficient system by modularizing the ASP programs and making domain-specific restrictions on the modules.  

Fortunately, we have many innovative ideas for improving the performance of such solvers by several orders of magnitude. Our ideas are derived from the state-of-the-art techniques developed by our teammate for solving Weighted Constraint Satisfaction Problems (WCSPs). 
% The success of most solvers for basic ASP is derived from that of SAT solvers. As a consequence, 
The WCSP serves as a substrate combinatorial problem for ASP augmented with probabilities and weights. In addition, the WCSP is representationally powerful enough to capture many other combinatorial problems in probabilistic reasoning. We have developed the state-of-the-art methods for solving WCSPs. We can apply these techniques to solve ASP augmented with probabilities and weights. Our intuitions include:
\begin{itemize}
\item The Constraint Composite Graph (CCG)~\cite{k08,k08a,k16} is a graph associated with a WCSP. It can be constructed in polynomial time.  Solving a WCSP is equivalent to solving the Minimum Weighted Vertex Cover (MWVC) problem on its CCG. 
% The CCG provides the first unifying mathematical framework for simultaneously exploiting the locality of interactions between the variables as well as the exact probabilities and weights in these interactions~\cite{}. Sufficient structure in locality of interactions translates to low tree width of the CCG; and sufficient structure in the probabilities and weights translates to bipartivity of the CCG.
\item The {\em half-integral\/} property of the MWVC problem can be exploited towards a kernelization procedure for the WCSP~\cite{cc08}. We can stage a polynomial-time maxflow algorithm that preprocesses a WCSP and fixes the optimal values of a large number of its variables. The remaining variables constitute the {\em core\/} of the combinatorial problem. 
% Our polynomial-time kernelization procedure is so powerful that on about $1/7^{th}$ of the WCSP benchmark instances, the core turns out to be empty, i.e., the problem instances are solved entirely in the polynomial-time phase of the algorithm without requiring search at all~\cite{}. In general, for the remaining core of the combinatorial problem, we have also developed the state-of-the-art MWVC solvers~\cite{}.
\item Message Passing (MP) is a well known technique used for large scale combinatorial problems~\cite{kf09}. It avoids exponential time complexity and constitutes an anytime algorithm.  It is known to work very well on several inference problems in Information Theory~\cite{yfw03}.
% However, its behavior was not well understood on Probabilistic Graphical Models or WCSPs until recently. In our recent work, 
We have shown that MP applied on the CCG of a WCSP produces solutions of much higher quality compared to MP applied directly on the WCSP~\cite{xu2017}.
\end{itemize}

% The recent ASP competition provides benchmark performance \cite{ASPC6}.  With a fixed problem structure in our DAC it is unlikely the multiple portfolio (algorithm selection) technique employed by ME-ASP would add any benefit for its implementation complexity, perhaps favoring the simpler WASP+DLV approach.  

ASP in general is an NP-complete problem. The stratified subset of ASP is quadratic. By having a modular language that separates the specification to a small non-stratified part and having a specialized solver for stratified ASP, we expect order of magnitude improvement in performance.  A recent work~\cite{cuteri} with similar ideas has shown good initial results. In addition, grounding can be done faster using a parallelized system in a multi-core environment.

Recent ASP competition results \cite{ASPC6} note the apparent performance benefit of domain modeling and specialization.  % They offer ``[t]hese advantages are owed to symmetry breaking, cutting down a vast number of redundant representations of solution candidates, as well as compact formulations, reducing the size of ground instantiations by some order of magnitude.'' 
Because we develop a paired DSL and solver implementation, such techniques are applicable to us for scalability enhancements.  The competition revealed that on decision problems like ours, the fastest general-purpose solvers produced solutions at a rate of $0.2$ problems per second on the reference Linux platforms.  We expect our specialized solvers to well exceed this performance level on our target application.

% the Multi-Engine ASP Solver (ME-ASP) \cite{MEASP} and WASP+DLV technologies performed the best in the competition's single-processor track.  ME-ASP average time on an array of grounded decision and query problems amounted to $0.2$ problems per second on the reference Linux platform.  However, ME-ASP uses algorithm selection to provide general-purpose coverage; our specialized solver should be expected to exceed this performance threshold on its target application.  

\paragraph{Integration and Demonstration}

We envision providing an API for content producers and consumers to use for interacting with the DAC.  Our design and verification tools and run-time monitors and operators will assert assurance argument structures and features by way of the API.  Any DAC control parameters will be done with a separate API.  We envision using a representational state transfer (REST) stateless specification and synthesizing software adapters for this API.  

As an example of high-level safety requirements, we would commence with explicit control system safety claims in our design activity.  A representative set is given in \cite{NPR7150} regarding completeness of requirements for safety-critical software.  These address conditions such as reachability of safe and unsafe states, operator overrides, error handling, and so on.  We will assert these as specific claims under a top-level safety claim for the assurance argument, then propagate the claims through our hierarchical design and hazard analysis procedure.  
% Various levels of abstraction or refinement of the models provide content for the argument, contributing various granularities of claim and evidence for composing the argument.   
% In some cases the design and verification steps will produce corresponding run-time monitors for conditional evidence gathering.  
Verification tools produce monitors to cover the behaviors of the system that could not otherwise be proven correct -- for which more information is needed.  We determine and justify the need for the run-time monitor and its conditional evidence as part of the design.  
Our DAC reasoning engine may be designed in as an integral part of the safety framework, thus having safety requirements of its own in the integrated system.  We will provide all integration support via software installation and API services, and support demonstration planning and conduct as requested. 




\begin{table}
\caption{Project Plan for TA3}
  \centering
  {\footnotesize
\begin{tabular}{|m{.6in}|m{5.55in}|} 
\hline
\textbf{Phase} & \textbf{Plan} 
\\\hline
Phase I & 
We will build the correspondence mechanisms for the safety-verification-monitor checks across the assurance case, focusing on contracts and constraints at various levels of abstraction in the design and implementation.  We update an existing DSL and ASP solver to focus on this LE-CPS application, and build an API for exchanging constructs and data with the DAC.  We show how the reasoning unfolds for successful and unsuccessful events.   
\\ \hline
Phase II & 
We will elaborate on our Phase I activities to include compositional arguments, uncertainty representations, learning-enabled ASP solvers, adjusting DSL, API, and solvers as required for successful demonstration.  We anticipate having more formal verification and monitor data available at different levels of granularity, both with impact to the DSL. Initial efforts for ASP solver speed scale-up will be undertaken.  
\\ \hline
Phase III & 
We will emphasize scale-up with a primary concern for speed. We will implement modular solvers tailored to our domain, runtime environment, and operating constraints. We  anticipate having more formal verification data available from TA1 at different levels of granularity, as well as more monitor types available from TA2, both with impact to the DSL.
\\ \hline
\end{tabular}
}
\label{tab:ta3:plan}
\end{table}

% During Phase I we will build the correspondence mechanisms for the safety-verification-monitor checks across the assurance case, and show how the case reasoning unfolds for both successful and unsuccessful events.   During Phase II we will expand these techniques to compositional arguments for multiple components acting together to comprise the system.   During Phase III we will further expand to include static analysis correspondence between source and binary images, and including results from dynamic analysis for other properties of interest to the assurance case.  


% note use of enumitem package and [leftmargin=*] to eliminate itemize indentation
\begin{table}
\caption{How TA3 Interfaces with TA1 and TA2}
  \centering
  {\footnotesize
\begin{tabular}{|m{.5in}|m{5.65in}|} 
\hline
\textbf{TA3 } & \textbf{Interfaces and Interaction} 
\\\hline
 TA1 & 
\begin{itemize}[leftmargin=*,topsep=3pt]
\setlength\itemsep{0em} 
%\setlength\parskip{0em}
%\setlength\parsep{0em}
\item Receive assurance-related content in the form of contracts, assumptions, and safety constraints.  Additional information such as  bounds for any learning conducted within the assurance case itself.  Much of this content becomes claims for the assurance case argument.   
\item Receive content in the form of property specifications, open and closed proof obligations, assumptions, interactions, and constraints that led to those results.  
\item Specify and implement an API for the assurance platform which accepts formal verification, simulation, and system testing claims along with assurance case. 
\end{itemize}
\\ \hline
TA2 & 
\begin{itemize}[leftmargin=*,topsep=3pt]
\setlength\itemsep{0em}
\item To support moving evidence into the DAC, TA3 will specify and implement an API for the assurance platform that can receive evidence, conditional evidence, and supporting data.  
\item Evaluate DACs and provide assurance measure when conditional evidence is received. 
\item To support visualization and logging, TA3 will provide an API and services for run-time monitoring of the DAC.
\item To support operational integrity, the design may require fault tolerance for DAC services.  TA3 will provide features for checkpoints, redundancy, roll-backs, or other integrity features. 
\end{itemize}
\\\cline{1-2}
\hline
\end{tabular}
}
\label{tab:ta3:if}
\end{table}




\begin{table}
\caption{Technical Challenges and Mitigation Approach for TA3}
  \centering 
{\footnotesize
\begin{tabular}{|m{1.3in}|m{4.85in}|} 
\hline
\textbf{Challenge} & \textbf{Mitigation Approach} 
\\\hline

Hazard analysis data volume  & 
The ASP technology we use for DAC can easily accommodate large volumes of data.  The ASP reasoning technique enables filtering the results into easily contrasted outcomes so that the practitioner can identify conflicts.\\ \hline

DAC design and implementation validity & 
We consider validation from four perspectives: {\em construct validity\/} (on the theory of DAC and its implementation), {\em conclusion validity\/} (claims to evidence), {\em  internal validity\/} (causal or logical relationships within an argument), and {\em external validity\/} (generalization of results).  
\\ \hline

Identifying sources of verification errors &
Potential sources of verification errors include: modeling errors; testing harnesses and stubbed behaviors; property specification errors; incomplete test results; interpretation (assumption) errors; soundness vs. completeness inference errors in formal methods; artifact misinterpretation or corruption; and tool configuration managements and operation errors. To mitigate these risks we impose assurance arguments on our tool chains.  
\\ \hline

ASP solver speed improvement by specialization to application &
Some expectation of availability and reliability of the DAC application will be imposed by the runtime environment. The challenges are accelerating the ASP solvers and handling time synchronization across multiple DAC instances.
% ; handling fresh and stale data mix; data service drops or corruption; handling defaults in the absence of data; recovery after stop; managing human intervention; and service degradations. 
We explore multiple forms of modular and potentially translated implementations.  
We incorporate the DAC into the system design, potentially including hard real-time and safety downmode features. \\ \hline
\end{tabular}
}
\label{tab:ta3:risk}
\end{table}






\subsection{Dynamic Assurance}

\subsubsection{Current Approaches and Issues} 
% Suggested relocation color code: 
%\textcolor{orange}{Move to Overview}
%\textcolor{blue}{Move to TA1}
%\textcolor{cyan}{Move to TA2}

%\begin{itemize}
%\item Briefly describe current approaches
%\item Discuss challenges/issues with these approaches
%\item What makes our approach innovative with respect to the state-of-the-art
%\end{itemize}

% \subsubsection{Current Approaches and Issues}

A recent survey of safety engineering compliance practices relates to our dynamic assurance case  challenge (\cite{RESSJ2013}).  The survey found that practitioners frequently relied upon {\em process-based evidence}, such as verification and validation plans and on safety management plans. When the practitioners provided {\em product-based\/} evidence they provided the requirements specification and test results. The study authors mention their intrigue that evidence types concerning risks and hazards are not among the most frequently reported. 
% Practitioners most often use checklists and expert judgment (with rationale) for assessing evidence.  
Their highest challenge is determining the confidence in evidence supporting a claim about system safety. 
% A critical challenge is gathering the design and development artifacts along with the decision process involved to collect them.  
Formal verification results were used by fewer than 30\% of the study respondents.  Completeness of the evidence is checked {\em manually\/} and through a manual predefined process, according to more than half of the respondents. 

Regarding the structure of an assurance case, it has been popular for many years to refer to legal reasoning and graphical notations (\cite{kelly04}).  
% For small arguments, these graphical notations illustrate how to go about structuring and supporting an argument for the trustworthiness of the system.  
The tree-structured {\em goal structuring notation\/} (GSN) is popular, and its cousin dubbed the {\em claim-argument-evidence\/} is an alternate representation.  
% Again for small arguments, one could question whether these are an improvement over free-form text.  
For larger arguments, the tree structures quickly become undesirable, fine for computer internal representation but clumsy for human-computer interfaces or printed documents.  Moreover, there is no specification for the automated analysis of these trees, leaving the user to impose writing requirements to facilitate analysis (\cite{rushby10}).  Somewhere in-between graphical notations and free-form text is semi-structured text (\cite{CMK08}), a DSL for writing case arguments while lending support for fancy editors and automated analysis.

Interesting assurance case structures that lend themselves to analysis have been developed.  
% We describe here the Resolute language, the Evidential Tool Bus, and the $L$ domain specific language (DSL).  
The Resolute language consists of both a logic and a computational sublanguage (\cite{Resolute}). The logic of Resolute is an intuitionistic logic similar to pure Prolog, but augmented with explicit quantification. 
% The logic is parameterized by the computational sublanguage, and requires only that the sub-language is deterministic. 
% This allows the computational sublanguage to be customized to any domain, such as AADL, and to be expanded and refined, without worrying about the logical consequences.   
The Evidential Tool Bus (ETB) (\cite{ETB}) is similar in syntax and semantics to Resolute. It provides a Datalog-style logic and is designed to combine evidence from a variety of sources. However, the focus of the ETB is on distribution and on provenance.  
% It logs the sequence of tool invocations that were performed to solve a query. 
% It uses timestamps to determine which analyses are out of date with respect to the current development artifacts and to only re-run those analyses that are not synchronized with the current development artifacts. 
% ETB is meant to be tool and model agnostic, and therefore has no particular knowledge of the system architecture.  
The DSL $L$ takes a semi-structured text approach (\cite{CertWareABSA}).  It provides an English-like language having a reasonable syntax and semantics for authors to write cases in an argument structure, but also provides analytic support for the document.  The $L$ tool set provides an answer set solver to reason about the case argument and return results to the author. 


% \subsubsection{Issues with Current Approaches}

The safety practitioner survey revealed that simple textual templates are the most frequently found technique for evidence structuring. Study respondents reported rarely using argumentation-based graphical notation such as goal-structuring notation (GSN) or process models such as the OMG's software and system process engineering metamodel (SPEM).  Regarding argumentation-based graphical notations, the study authors note ``[t]he results suggest that a lot of research effort has been spent on a technique that has seen little industrial adoption thus far''.  We speculate, based on our own experience building tool sets for such graphical models, that the problem of quickly becoming unwieldy when modeling real-world products undermines the appeal of these techniques. 

The notation problem can be addressed by domain-specific languages that enable analysis support.  We introduced {\em answer set programming\/} (ASP) technology to provide this support \cite{CertWareABSA}, illustrating analysis of arguments, but the implementation lacks strong formalisms for handling learning, reasoning about uncertainty, and performance scalability.  Although there are some existing algorithms for learning assurance rules in ASP \cite{ray2009nonmonotonic,athakravi2013learning,law2014inductive,athakravi2015inductive,iled,kazmi2017improving}, they do not work well with large data sets. The best performing among them is the system XHAIL.  For handling uncertainty popular formalisms such as Markov Logic Networks or its recently proposed ASP variant $LP^{MLN}$ do not provide for non-monotonic specification of randomness.  An ASP-specific variant P-log does have the required construct, but lacks in its ability to specify arbitrary weights.  

The LE components introduce unique issues for safety assurance~\cite{SafetyANN}. For example, neural networks lend themselves only to black box analysis but are wanted under verification and validation regimes that require white box analysis.  Hazard analysis cannot be performed in the usual manner owing to the way the network algorithms work.  Implicit, process-based safety assurance is insufficient for neural networks owing to a lack of concern for the functional behaviors of the software.  Explicit, product-based arguments such as proposed herein are more appropriate to consider the functional behaviors. 
% To support the top-level assurance claim for an ANN component we must provide the context for the claim and its support.  The context defines characteristics of the model, appropriate application for the design, and a definition of an {\em acceptably safe\/} threshold. 




\subsubsection{Approach}

\paragraph{Domain Specific Language}

We refer to assurance cases having the conventional structure of a legal argument justifying trustworthiness for the system, in a {\em claim-argument-evidence motif}.  We populate the assurance case differently from most in that we do not use the conventional tree structures.  
% Instead we use declarative observations gathered from the physical system and its requirements,  assumptions, and constraints.  
Our declarative approach employs a DSL as semi-structured text that lends itself to automated analysis.  
We use the design and verification content to populate the assurance case argument for safety property claims and evidence at build-time, coupled with run-time monitors produced alongside the evidence to be checked during operation.  Both the build-time evidence and the production evidence in the learning-enabled environment integrate with the dynamic assurance case to maintain the argument for the trustworthiness of the system.  

We will enhance an existing assurance case DSL called $L$. The DSL works in conjunction with an answer set program (ASP) solver to check assurance case arguments.  We choose ASP technology here over other solvers such as SAT or SMT because, after grounding or translation, it provides better access modeling and specialization opportunities for domain-specific scalability.  When the dynamic argument solution shows that trustworthiness can no longer be established, the the answer set solution will show the conflict by way of admitting multiple models.   We bring the language, the editors, and solvers forward as our starting point for our dynamic assurance case (DAC) implementation.

Leveraging our systems-theoretic approach to design and implementation, we reason about the essential elements of {\em intent\/} and {\em state\/} in the DAC.  Our DAC focuses on reasoning about the states of the system because these states are essential for understanding and managing the constraints imposed for LE-CPS safety.  Moreover, through the DAC we adorn the states with associated design contracts, safety constraints, assumptions, intentions, and interactions among the states.  
% Each state is a template consisting of a state variable history, an estimator, a controller, and if interacting with the world a hardware adapter. While normally such a template facilitates methodical design activities, we adopt it for DAC elements as well.  
For human interaction and debugging, when we provide system design intent in terms of state we can use this information from the DAC to support diagnostic explanations.  The current $L$ language resembles a blend of English and Prolog, and we provide editor support to emphasize ease of use.  


% As an example consider a sample assurance case fragment (Figure~\ref{fig:ta3:tc}) from an $L$ argument for design tool chain verification with the goal of claiming certification credit~\cite{FACTS16}.  The case author proceeds from a high-level risk specification ({\tt invalidity\-Risk}) to refine the many instances (e.g. {\tt mitigated} and {\tt verification\-Risk}) involved in the argument.  
% \begin{figure}[tbp]
% \lstset{language=Prolog}
% \begin{lstlisting}
% invalidityRisk(certification(property P, software S)) if
% 	reliesOn(certification(P,S), evidence E) and 
%     produces(toolChain TC, E, P, S) and
%     verificationRisk(risk R, P, S, TC) and
%     not mitigated(R, P, S, TC).
    
% verificationRisk(unsound(modelChecker), property P, software S, toolChain T) if
% 	modelOf(S, model M) and
%     executes(modelChecker MC, input(P, M), output(verified), T) and
%     produces(T, evidence E, P, S) and
%     includes(E, verified(P, M, MC)).
    
% mitigated(unsound(modelChecker), property P, software S, toolChain T) if
% 	modelOf(S, model M) and
%     executes(modelChecker MC1, input(P, M), output(verified), T) and
%     executes(modelChecker MC2, input(P, M), output(verified), T) and
%     MC1 != MC2.

% type modelChecker = {modelChecker1, modelChecker2}.
% type risk = {unsound(modelChecker)}. 
% ...
% \end{lstlisting}
% \caption[Fragment listing of $L$ tool chain argument]{A fragment assurance case listing written in the $L$ domain-specific language. }
% \label{fig:ta3:tc}
% \end{figure}

\begin{wrapfigure}{R}{0.6\textwidth}
\vspace{-30pt}
% \begin{figure}[tbhp]
\begin{center}
% \includegraphics[width=0.8\textwidth]{system_model}
\includegraphics[width=0.6\textwidth,trim={0in 0.2in 0in 0in},clip]{./TA3/dac-sources}
\end{center}
\caption[Assurance case sources]{The system design and runtime activities provide different kinds of content, at different levels of abstraction or refinement, to the dynamic assurance case. }
% \caption[Assurance case sources]{System safety models at different levels of abstraction.  For the state model, instantiations of the pattern for each system state help to specify the system models and to communicate system safety data to the assurance case.  For the system model, control loops around a controlled process help capture the design constraints for the DAC.  } 
\label{fig:statemodel}
% \end{figure}
\vspace{-25pt}
\end{wrapfigure}

\paragraph{Assurance Solver Implementations}

We will improve one or more reasoning engines for our DAC DSL to improve their suitability for dynamic assurance.  There exists several excellent inference engines for basic ASP. 
These include Clingo~\cite{clingo} and Potassco~\cite{potassco}, smodels~\cite{smodels}, and DLV for basic ASP and some systems for languages such as P-log and $LP^{MLN}$ that augment ASP with probabilities and weights. We anticipate building specialized solvers, rather than general-purpose solvers, to support our goals for the program. 

\noindent\underline{Supporting Learning}

During Phase I, we will modify $L$ to accommodate learning of assurance rules.  To have a common formal representation of various modules and components of the learning enabled autonomous systems, we aim to develop learning enabled components that can be characterized using the common representation. 
% For a seamless integration we aim to develop tools and algorithms that can learn rules in ASP and $L$ from data. 

The general sub-area of learning logic rules from data is called inductive logic programming (ILP). 
In statistical machine learning, it is common to learn a function from a series of $\langle x, y \rangle$ pairs where $x$ denotes the input and $y$ denotes the desired output. On the other hand, ILP deals with learning a logic program $H$ given some background knowledge and a dataset containing positive and negative examples. More formally, Given a set of positive examples $E^{+}$, negative examples $E^{−}$ and some background knowledge $B$, an ILP algorithm \cite{muggleton1991inductive} finds a hypothesis $H$ such that, $B \cup H  \models E^{+}$, $B \cup H \not \models E^{−}$. The possible hypothesis space is often restricted with a language bias that is specified by a series of mode declarations $M$.

Because of the mismatch between statistical machine learning and ILP definitions, one needs to convert all the $\langle x,y \rangle$ to create a global $E^{+}$ and $E^{−}$ and extra care needs to be taken so that different $\langle x,y \rangle$   pairs do not interfere with each other.  This approach fails to scale up. That is because, when we have a large  number of $\langle x,y \rangle$  pairs, the ILP solvers will then have to deal with large ASP programs,  We propose to develop a novel iterative and incremental approach where instead of making the above conversion, the learning of rules is done by iteratively going over each example $\langle x,y \rangle$ pair.   Our approach is based on {\em context dependent learning}. By changing the inductive logic programming paradigm to context dependent learning, we will be avoiding dealing with large examples that is made of combining the set of examples. This allows us to pursue an iterative and incremental approach making our approach scalable.  We expect an order of magnitude or more improvement. 
% In this formulation, each example $E_i$ directly corresponds to an $\langle x,y \rangle$ pair and it takes into account that there are several distinct examples in a dataset. 
%We aim to develop an algorithm that will find the solutions incrementally. First it will compute the solutions of $ILP^{ctx}(\langle B,M,{E_1}\rangle)$, then it will use those to find the solutions $ILP^{ctx}(\langle B,M,{E_1, E_2} \rangle)$ and so on until it finds a solution for the original problem, $ILP^{ctx}(\langle B,M,{E_1,...E_n}\rangle)$. Thus, our algorithm will not compute the solutions of $ILP^{ctx}(\langle B,M,{E_1, E_2,..E_i}\rangle)$ in a single step. Rather it will compute several lower bound-upper bound pairs on the search space and employ a state space search for each lower bound-upper bound pair until all the minimal solutions in that space are found. 


% , defined as follows:
%{\bf Definition}:	A \textit{Context Dependent Learning} task $ILP^{ctx}$ is a tuple $\langle B,M,D \rangle$, where $B$ is an Answer Set Program, called the background knowledge, $M$ defines the set of rules allowed in hypotheses (the hypothesis space) and $D$ is the dataset containing a series of context dependent examples $E_1,E_2,..., E_n$. Here each $E_i$ is a tuple $\langle O_i,E_{i}^{+}, E_{i}^{-} \rangle$ where,  $O_i$ is a logic program, called \textit{observation} , $E^{+}$ is a set of positive ground literals and $E^{-}$ is a set of negative ground literals. A hypothesis $H$ is an inductive solution of $T$ (written as $H \in ILP^{ctx}(B,M,D)$) \textit{iff}
%	$H\cup B \cup O_i \vdash E_{i}^{+}, ~\forall i=1...n$,  and   
%	$H\cup B \cup O_i \not \vdash E_{i}^{-}, ~\forall i=1...n$. 

Another learning aspect we will address is allowing previously undefined predicates in $H$. Currently, in all ILP systems and algorithms the mode definitions of rules in $H$ do not allow previously undefined predicates. 
% However, such predicates are needed to learn rules whose conditional part may contain universal quantifications. 
The universal quantification plays an important role in the specification of temporal operators ``{\em always\/}'' and ``{\em until\/}''.

\noindent\underline{Supporting Uncertainty}

During Phase II, we will add support for uncertainty. 
%In recent years several extensions of ASP have been proposed to account for uncertainty.  One approach, as used in P-log, enhances ASP with some explicit probability assignments and additional implicitly derived probability numbers. In another approach, as used in LP\^MLN (and even in the much earlier introduced ASP with weak constraints), weight assignments are given explicitly and probability numbers are computed on top of them.  
We propose to develop a new formalism that incorporates reasoning features crucial to assurance with respect to autonomous systems, such as causality, counter-factual reasoning, use of weights for computing probabilities, 
% instead of direct probability numbers, which are hard to estimate for assurance rules) from which probabilities are computed, 
and probabilistic non-monotonicity -- the ability to have new possible models as a result of new information. 
Probabilistic non-monotonicity is useful for non-\naive\ conditioning, as \naive\ conditioning may fail with new sensors that ask different kind of questions and as a result need a new set of possible worlds \cite{halpern2003, baral2007}. 
% (Such a situation is described in \cite{halpern2003} and addressed in the context of ASP by us in \cite{baral2007}.) 

%To allow non-\naive\ conditioning we need rules of the form:
%\begin{equation*}
%{\tt random}(o(Y) : \{X : p(X)\}) :- a_1, \ldots , a_n, \ {\bf not} \ b_1, \ldots, \ {\bf not} \ b_m. 
%\end{equation*}
%The above rules state that $o(Y)$ is a random variable where $Y$ can take the value from the set \{X : p(X)\} when $B$ is true. Such a construct allows non-monotonic specification of when $o(Y)$ is a random variable and what its extent is. Such non-monotonic specification of randomness is not possible in the popular uncertainty formalisms such as Markov Logic Networks or its recently proposed ASP variant $LP^{MLN}$. While P-log has such a construct, it lacks in its ability to specify arbitrary weights. 

\noindent\underline{Supporting Scaling}

During Phase III, we will focus on ASP solver speed improvements.  There exist excellent inference engines for basic ASP, but none of them has emphasized speed.  We will develop a more efficient system by modularizing the ASP programs and making domain-specific restrictions on the modules.  

Fortunately, we have many innovative ideas for improving the performance of such solvers by several orders of magnitude. Our ideas are derived from the state-of-the-art techniques developed by our teammate for solving Weighted Constraint Satisfaction Problems (WCSPs). 
% The success of most solvers for basic ASP is derived from that of SAT solvers. As a consequence, 
The WCSP serves as a substrate combinatorial problem for ASP augmented with probabilities and weights. In addition, the WCSP is representationally powerful enough to capture many other combinatorial problems in probabilistic reasoning. We have developed the state-of-the-art methods for solving WCSPs. We can apply these techniques to solve ASP augmented with probabilities and weights. Our intuitions include:
\begin{itemize}
\item The Constraint Composite Graph (CCG)~\cite{k08,k08a,k16} is a graph associated with a WCSP. It can be constructed in polynomial time.  Solving a WCSP is equivalent to solving the Minimum Weighted Vertex Cover (MWVC) problem on its CCG. 
% The CCG provides the first unifying mathematical framework for simultaneously exploiting the locality of interactions between the variables as well as the exact probabilities and weights in these interactions~\cite{}. Sufficient structure in locality of interactions translates to low tree width of the CCG; and sufficient structure in the probabilities and weights translates to bipartivity of the CCG.
\item The {\em half-integral\/} property of the MWVC problem can be exploited towards a kernelization procedure for the WCSP~\cite{cc08}. We can stage a polynomial-time maxflow algorithm that preprocesses a WCSP and fixes the optimal values of a large number of its variables. The remaining variables constitute the {\em core\/} of the combinatorial problem. 
% Our polynomial-time kernelization procedure is so powerful that on about $1/7^{th}$ of the WCSP benchmark instances, the core turns out to be empty, i.e., the problem instances are solved entirely in the polynomial-time phase of the algorithm without requiring search at all~\cite{}. In general, for the remaining core of the combinatorial problem, we have also developed the state-of-the-art MWVC solvers~\cite{}.
\item Message Passing (MP) is a well known technique used for large scale combinatorial problems~\cite{kf09}. It avoids exponential time complexity and constitutes an anytime algorithm.  It is known to work very well on several inference problems in Information Theory~\cite{yfw03}.
% However, its behavior was not well understood on Probabilistic Graphical Models or WCSPs until recently. In our recent work, 
We have shown that MP applied on the CCG of a WCSP produces solutions of much higher quality compared to MP applied directly on the WCSP~\cite{xu2017}.
\end{itemize}

% The recent ASP competition provides benchmark performance \cite{ASPC6}.  With a fixed problem structure in our DAC it is unlikely the multiple portfolio (algorithm selection) technique employed by ME-ASP would add any benefit for its implementation complexity, perhaps favoring the simpler WASP+DLV approach.  

ASP in general is an NP-complete problem. The stratified subset of ASP is quadratic. By having a modular language that separates the specification to a small non-stratified part and having a specialized solver for stratified ASP, we expect order of magnitude improvement in performance.  A recent work~\cite{cuteri} with similar ideas has shown good initial results. In addition, grounding can be done faster using a parallelized system in a multi-core environment.

Recent ASP competition results \cite{ASPC6} note the apparent performance benefit of domain modeling and specialization.  % They offer ``[t]hese advantages are owed to symmetry breaking, cutting down a vast number of redundant representations of solution candidates, as well as compact formulations, reducing the size of ground instantiations by some order of magnitude.'' 
Because we develop a paired DSL and solver implementation, such techniques are applicable to us for scalability enhancements.  The competition revealed that on decision problems like ours, the fastest general-purpose solvers produced solutions at a rate of $0.2$ problems per second on the reference Linux platforms.  We expect our specialized solvers to well exceed this performance level on our target application.

% the Multi-Engine ASP Solver (ME-ASP) \cite{MEASP} and WASP+DLV technologies performed the best in the competition's single-processor track.  ME-ASP average time on an array of grounded decision and query problems amounted to $0.2$ problems per second on the reference Linux platform.  However, ME-ASP uses algorithm selection to provide general-purpose coverage; our specialized solver should be expected to exceed this performance threshold on its target application.  

\paragraph{Integration and Demonstration}

We envision providing an API for content producers and consumers to use for interacting with the DAC.  Our design and verification tools and run-time monitors and operators will assert assurance argument structures and features by way of the API.  Any DAC control parameters will be done with a separate API.  We envision using a representational state transfer (REST) stateless specification and synthesizing software adapters for this API.  

As an example of high-level safety requirements, we would commence with explicit control system safety claims in our design activity.  A representative set is given in \cite{NPR7150} regarding completeness of requirements for safety-critical software.  These address conditions such as reachability of safe and unsafe states, operator overrides, error handling, and so on.  We will assert these as specific claims under a top-level safety claim for the assurance argument, then propagate the claims through our hierarchical design and hazard analysis procedure.  
% Various levels of abstraction or refinement of the models provide content for the argument, contributing various granularities of claim and evidence for composing the argument.   
% In some cases the design and verification steps will produce corresponding run-time monitors for conditional evidence gathering.  
Verification tools produce monitors to cover the behaviors of the system that could not otherwise be proven correct -- for which more information is needed.  We determine and justify the need for the run-time monitor and its conditional evidence as part of the design.  
Our DAC reasoning engine may be designed in as an integral part of the safety framework, thus having safety requirements of its own in the integrated system.  We will provide all integration support via software installation and API services, and support demonstration planning and conduct as requested. 




\begin{table}
\caption{Project Plan for TA3}
  \centering
  {\footnotesize
\begin{tabular}{|m{.6in}|m{5.55in}|} 
\hline
\textbf{Phase} & \textbf{Plan} 
\\\hline
Phase I & 
We will build the correspondence mechanisms for the safety-verification-monitor checks across the assurance case, focusing on contracts and constraints at various levels of abstraction in the design and implementation.  We update an existing DSL and ASP solver to focus on this LE-CPS application, and build an API for exchanging constructs and data with the DAC.  We show how the reasoning unfolds for successful and unsuccessful events.   
\\ \hline
Phase II & 
We will elaborate on our Phase I activities to include compositional arguments, uncertainty representations, learning-enabled ASP solvers, adjusting DSL, API, and solvers as required for successful demonstration.  We anticipate having more formal verification and monitor data available at different levels of granularity, both with impact to the DSL. Initial efforts for ASP solver speed scale-up will be undertaken.  
\\ \hline
Phase III & 
We will emphasize scale-up with a primary concern for speed. We will implement modular solvers tailored to our domain, runtime environment, and operating constraints. We  anticipate having more formal verification data available from TA1 at different levels of granularity, as well as more monitor types available from TA2, both with impact to the DSL.
\\ \hline
\end{tabular}
}
\label{tab:ta3:plan}
\end{table}

% During Phase I we will build the correspondence mechanisms for the safety-verification-monitor checks across the assurance case, and show how the case reasoning unfolds for both successful and unsuccessful events.   During Phase II we will expand these techniques to compositional arguments for multiple components acting together to comprise the system.   During Phase III we will further expand to include static analysis correspondence between source and binary images, and including results from dynamic analysis for other properties of interest to the assurance case.  


% note use of enumitem package and [leftmargin=*] to eliminate itemize indentation
\begin{table}
\caption{How TA3 Interfaces with TA1 and TA2}
  \centering
  {\footnotesize
\begin{tabular}{|m{.5in}|m{5.65in}|} 
\hline
\textbf{TA3 } & \textbf{Interfaces and Interaction} 
\\\hline
 TA1 & 
\begin{itemize}[leftmargin=*,topsep=3pt]
\setlength\itemsep{0em} 
%\setlength\parskip{0em}
%\setlength\parsep{0em}
\item Receive assurance-related content in the form of contracts, assumptions, and safety constraints.  Additional information such as  bounds for any learning conducted within the assurance case itself.  Much of this content becomes claims for the assurance case argument.   
\item Receive content in the form of property specifications, open and closed proof obligations, assumptions, interactions, and constraints that led to those results.  
\item Specify and implement an API for the assurance platform which accepts formal verification, simulation, and system testing claims along with assurance case. 
\end{itemize}
\\ \hline
TA2 & 
\begin{itemize}[leftmargin=*,topsep=3pt]
\setlength\itemsep{0em}
\item To support moving evidence into the DAC, TA3 will specify and implement an API for the assurance platform that can receive evidence, conditional evidence, and supporting data.  
\item Evaluate DACs and provide assurance measure when conditional evidence is received. 
\item To support visualization and logging, TA3 will provide an API and services for run-time monitoring of the DAC.
\item To support operational integrity, the design may require fault tolerance for DAC services.  TA3 will provide features for checkpoints, redundancy, roll-backs, or other integrity features. 
\end{itemize}
\\\cline{1-2}
\hline
\end{tabular}
}
\label{tab:ta3:if}
\end{table}




\begin{table}
\caption{Technical Challenges and Mitigation Approach for TA3}
  \centering 
{\footnotesize
\begin{tabular}{|m{1.3in}|m{4.85in}|} 
\hline
\textbf{Challenge} & \textbf{Mitigation Approach} 
\\\hline

Hazard analysis data volume  & 
The ASP technology we use for DAC can easily accommodate large volumes of data.  The ASP reasoning technique enables filtering the results into easily contrasted outcomes so that the practitioner can identify conflicts.\\ \hline

DAC design and implementation validity & 
We consider validation from four perspectives: {\em construct validity\/} (on the theory of DAC and its implementation), {\em conclusion validity\/} (claims to evidence), {\em  internal validity\/} (causal or logical relationships within an argument), and {\em external validity\/} (generalization of results).  
\\ \hline

Identifying sources of verification errors &
Potential sources of verification errors include: modeling errors; testing harnesses and stubbed behaviors; property specification errors; incomplete test results; interpretation (assumption) errors; soundness vs. completeness inference errors in formal methods; artifact misinterpretation or corruption; and tool configuration managements and operation errors. To mitigate these risks we impose assurance arguments on our tool chains.  
\\ \hline

ASP solver speed improvement by specialization to application &
Some expectation of availability and reliability of the DAC application will be imposed by the runtime environment. The challenges are accelerating the ASP solvers and handling time synchronization across multiple DAC instances.
% ; handling fresh and stale data mix; data service drops or corruption; handling defaults in the absence of data; recovery after stop; managing human intervention; and service degradations. 
We explore multiple forms of modular and potentially translated implementations.  
We incorporate the DAC into the system design, potentially including hard real-time and safety downmode features. \\ \hline
\end{tabular}
}
\label{tab:ta3:risk}
\end{table}






\subsection{Dynamic Assurance}

\subsubsection{Current Approaches and Issues} 
% Suggested relocation color code: 
%\textcolor{orange}{Move to Overview}
%\textcolor{blue}{Move to TA1}
%\textcolor{cyan}{Move to TA2}

%\begin{itemize}
%\item Briefly describe current approaches
%\item Discuss challenges/issues with these approaches
%\item What makes our approach innovative with respect to the state-of-the-art
%\end{itemize}

% \subsubsection{Current Approaches and Issues}

A recent survey of safety engineering compliance practices relates to our dynamic assurance case  challenge (\cite{RESSJ2013}).  The survey found that practitioners frequently relied upon {\em process-based evidence}, such as verification and validation plans and on safety management plans. When the practitioners provided {\em product-based\/} evidence they provided the requirements specification and test results. The study authors mention their intrigue that evidence types concerning risks and hazards are not among the most frequently reported. 
% Practitioners most often use checklists and expert judgment (with rationale) for assessing evidence.  
Their highest challenge is determining the confidence in evidence supporting a claim about system safety. 
% A critical challenge is gathering the design and development artifacts along with the decision process involved to collect them.  
Formal verification results were used by fewer than 30\% of the study respondents.  Completeness of the evidence is checked {\em manually\/} and through a manual predefined process, according to more than half of the respondents. 

Regarding the structure of an assurance case, it has been popular for many years to refer to legal reasoning and graphical notations (\cite{kelly04}).  
% For small arguments, these graphical notations illustrate how to go about structuring and supporting an argument for the trustworthiness of the system.  
The tree-structured {\em goal structuring notation\/} (GSN) is popular, and its cousin dubbed the {\em claim-argument-evidence\/} is an alternate representation.  
% Again for small arguments, one could question whether these are an improvement over free-form text.  
For larger arguments, the tree structures quickly become undesirable, fine for computer internal representation but clumsy for human-computer interfaces or printed documents.  Moreover, there is no specification for the automated analysis of these trees, leaving the user to impose writing requirements to facilitate analysis (\cite{rushby10}).  Somewhere in-between graphical notations and free-form text is semi-structured text (\cite{CMK08}), a DSL for writing case arguments while lending support for fancy editors and automated analysis.

Interesting assurance case structures that lend themselves to analysis have been developed.  
% We describe here the Resolute language, the Evidential Tool Bus, and the $L$ domain specific language (DSL).  
The Resolute language consists of both a logic and a computational sublanguage (\cite{Resolute}). The logic of Resolute is an intuitionistic logic similar to pure Prolog, but augmented with explicit quantification. 
% The logic is parameterized by the computational sublanguage, and requires only that the sub-language is deterministic. 
% This allows the computational sublanguage to be customized to any domain, such as AADL, and to be expanded and refined, without worrying about the logical consequences.   
The Evidential Tool Bus (ETB) (\cite{ETB}) is similar in syntax and semantics to Resolute. It provides a Datalog-style logic and is designed to combine evidence from a variety of sources. However, the focus of the ETB is on distribution and on provenance.  
% It logs the sequence of tool invocations that were performed to solve a query. 
% It uses timestamps to determine which analyses are out of date with respect to the current development artifacts and to only re-run those analyses that are not synchronized with the current development artifacts. 
% ETB is meant to be tool and model agnostic, and therefore has no particular knowledge of the system architecture.  
The DSL $L$ takes a semi-structured text approach (\cite{CertWareABSA}).  It provides an English-like language having a reasonable syntax and semantics for authors to write cases in an argument structure, but also provides analytic support for the document.  The $L$ tool set provides an answer set solver to reason about the case argument and return results to the author. 


% \subsubsection{Issues with Current Approaches}

The safety practitioner survey revealed that simple textual templates are the most frequently found technique for evidence structuring. Study respondents reported rarely using argumentation-based graphical notation such as goal-structuring notation (GSN) or process models such as the OMG's software and system process engineering metamodel (SPEM).  Regarding argumentation-based graphical notations, the study authors note ``[t]he results suggest that a lot of research effort has been spent on a technique that has seen little industrial adoption thus far''.  We speculate, based on our own experience building tool sets for such graphical models, that the problem of quickly becoming unwieldy when modeling real-world products undermines the appeal of these techniques. 

The notation problem can be addressed by domain-specific languages that enable analysis support.  We introduced {\em answer set programming\/} (ASP) technology to provide this support \cite{CertWareABSA}, illustrating analysis of arguments, but the implementation lacks strong formalisms for handling learning, reasoning about uncertainty, and performance scalability.  Although there are some existing algorithms for learning assurance rules in ASP \cite{ray2009nonmonotonic,athakravi2013learning,law2014inductive,athakravi2015inductive,iled,kazmi2017improving}, they do not work well with large data sets. The best performing among them is the system XHAIL.  For handling uncertainty popular formalisms such as Markov Logic Networks or its recently proposed ASP variant $LP^{MLN}$ do not provide for non-monotonic specification of randomness.  An ASP-specific variant P-log does have the required construct, but lacks in its ability to specify arbitrary weights.  

The LE components introduce unique issues for safety assurance~\cite{SafetyANN}. For example, neural networks lend themselves only to black box analysis but are wanted under verification and validation regimes that require white box analysis.  Hazard analysis cannot be performed in the usual manner owing to the way the network algorithms work.  Implicit, process-based safety assurance is insufficient for neural networks owing to a lack of concern for the functional behaviors of the software.  Explicit, product-based arguments such as proposed herein are more appropriate to consider the functional behaviors. 
% To support the top-level assurance claim for an ANN component we must provide the context for the claim and its support.  The context defines characteristics of the model, appropriate application for the design, and a definition of an {\em acceptably safe\/} threshold. 




\subsubsection{Approach}

\paragraph{Domain Specific Language}

We refer to assurance cases having the conventional structure of a legal argument justifying trustworthiness for the system, in a {\em claim-argument-evidence motif}.  We populate the assurance case differently from most in that we do not use the conventional tree structures.  
% Instead we use declarative observations gathered from the physical system and its requirements,  assumptions, and constraints.  
Our declarative approach employs a DSL as semi-structured text that lends itself to automated analysis.  
We use the design and verification content to populate the assurance case argument for safety property claims and evidence at build-time, coupled with run-time monitors produced alongside the evidence to be checked during operation.  Both the build-time evidence and the production evidence in the learning-enabled environment integrate with the dynamic assurance case to maintain the argument for the trustworthiness of the system.  

We will enhance an existing assurance case DSL called $L$. The DSL works in conjunction with an answer set program (ASP) solver to check assurance case arguments.  We choose ASP technology here over other solvers such as SAT or SMT because, after grounding or translation, it provides better access modeling and specialization opportunities for domain-specific scalability.  When the dynamic argument solution shows that trustworthiness can no longer be established, the the answer set solution will show the conflict by way of admitting multiple models.   We bring the language, the editors, and solvers forward as our starting point for our dynamic assurance case (DAC) implementation.

Leveraging our systems-theoretic approach to design and implementation, we reason about the essential elements of {\em intent\/} and {\em state\/} in the DAC.  Our DAC focuses on reasoning about the states of the system because these states are essential for understanding and managing the constraints imposed for LE-CPS safety.  Moreover, through the DAC we adorn the states with associated design contracts, safety constraints, assumptions, intentions, and interactions among the states.  
% Each state is a template consisting of a state variable history, an estimator, a controller, and if interacting with the world a hardware adapter. While normally such a template facilitates methodical design activities, we adopt it for DAC elements as well.  
For human interaction and debugging, when we provide system design intent in terms of state we can use this information from the DAC to support diagnostic explanations.  The current $L$ language resembles a blend of English and Prolog, and we provide editor support to emphasize ease of use.  


% As an example consider a sample assurance case fragment (Figure~\ref{fig:ta3:tc}) from an $L$ argument for design tool chain verification with the goal of claiming certification credit~\cite{FACTS16}.  The case author proceeds from a high-level risk specification ({\tt invalidity\-Risk}) to refine the many instances (e.g. {\tt mitigated} and {\tt verification\-Risk}) involved in the argument.  
% \begin{figure}[tbp]
% \lstset{language=Prolog}
% \begin{lstlisting}
% invalidityRisk(certification(property P, software S)) if
% 	reliesOn(certification(P,S), evidence E) and 
%     produces(toolChain TC, E, P, S) and
%     verificationRisk(risk R, P, S, TC) and
%     not mitigated(R, P, S, TC).
    
% verificationRisk(unsound(modelChecker), property P, software S, toolChain T) if
% 	modelOf(S, model M) and
%     executes(modelChecker MC, input(P, M), output(verified), T) and
%     produces(T, evidence E, P, S) and
%     includes(E, verified(P, M, MC)).
    
% mitigated(unsound(modelChecker), property P, software S, toolChain T) if
% 	modelOf(S, model M) and
%     executes(modelChecker MC1, input(P, M), output(verified), T) and
%     executes(modelChecker MC2, input(P, M), output(verified), T) and
%     MC1 != MC2.

% type modelChecker = {modelChecker1, modelChecker2}.
% type risk = {unsound(modelChecker)}. 
% ...
% \end{lstlisting}
% \caption[Fragment listing of $L$ tool chain argument]{A fragment assurance case listing written in the $L$ domain-specific language. }
% \label{fig:ta3:tc}
% \end{figure}

\begin{wrapfigure}{R}{0.6\textwidth}
\vspace{-30pt}
% \begin{figure}[tbhp]
\begin{center}
% \includegraphics[width=0.8\textwidth]{system_model}
\includegraphics[width=0.6\textwidth,trim={0in 0.2in 0in 0in},clip]{./TA3/dac-sources}
\end{center}
\caption[Assurance case sources]{The system design and runtime activities provide different kinds of content, at different levels of abstraction or refinement, to the dynamic assurance case. }
% \caption[Assurance case sources]{System safety models at different levels of abstraction.  For the state model, instantiations of the pattern for each system state help to specify the system models and to communicate system safety data to the assurance case.  For the system model, control loops around a controlled process help capture the design constraints for the DAC.  } 
\label{fig:statemodel}
% \end{figure}
\vspace{-25pt}
\end{wrapfigure}

\paragraph{Assurance Solver Implementations}

We will improve one or more reasoning engines for our DAC DSL to improve their suitability for dynamic assurance.  There exists several excellent inference engines for basic ASP. 
These include Clingo~\cite{clingo} and Potassco~\cite{potassco}, smodels~\cite{smodels}, and DLV for basic ASP and some systems for languages such as P-log and $LP^{MLN}$ that augment ASP with probabilities and weights. We anticipate building specialized solvers, rather than general-purpose solvers, to support our goals for the program. 

\noindent\underline{Supporting Learning}

During Phase I, we will modify $L$ to accommodate learning of assurance rules.  To have a common formal representation of various modules and components of the learning enabled autonomous systems, we aim to develop learning enabled components that can be characterized using the common representation. 
% For a seamless integration we aim to develop tools and algorithms that can learn rules in ASP and $L$ from data. 

The general sub-area of learning logic rules from data is called inductive logic programming (ILP). 
In statistical machine learning, it is common to learn a function from a series of $\langle x, y \rangle$ pairs where $x$ denotes the input and $y$ denotes the desired output. On the other hand, ILP deals with learning a logic program $H$ given some background knowledge and a dataset containing positive and negative examples. More formally, Given a set of positive examples $E^{+}$, negative examples $E^{−}$ and some background knowledge $B$, an ILP algorithm \cite{muggleton1991inductive} finds a hypothesis $H$ such that, $B \cup H  \models E^{+}$, $B \cup H \not \models E^{−}$. The possible hypothesis space is often restricted with a language bias that is specified by a series of mode declarations $M$.

Because of the mismatch between statistical machine learning and ILP definitions, one needs to convert all the $\langle x,y \rangle$ to create a global $E^{+}$ and $E^{−}$ and extra care needs to be taken so that different $\langle x,y \rangle$   pairs do not interfere with each other.  This approach fails to scale up. That is because, when we have a large  number of $\langle x,y \rangle$  pairs, the ILP solvers will then have to deal with large ASP programs,  We propose to develop a novel iterative and incremental approach where instead of making the above conversion, the learning of rules is done by iteratively going over each example $\langle x,y \rangle$ pair.   Our approach is based on {\em context dependent learning}. By changing the inductive logic programming paradigm to context dependent learning, we will be avoiding dealing with large examples that is made of combining the set of examples. This allows us to pursue an iterative and incremental approach making our approach scalable.  We expect an order of magnitude or more improvement. 
% In this formulation, each example $E_i$ directly corresponds to an $\langle x,y \rangle$ pair and it takes into account that there are several distinct examples in a dataset. 
%We aim to develop an algorithm that will find the solutions incrementally. First it will compute the solutions of $ILP^{ctx}(\langle B,M,{E_1}\rangle)$, then it will use those to find the solutions $ILP^{ctx}(\langle B,M,{E_1, E_2} \rangle)$ and so on until it finds a solution for the original problem, $ILP^{ctx}(\langle B,M,{E_1,...E_n}\rangle)$. Thus, our algorithm will not compute the solutions of $ILP^{ctx}(\langle B,M,{E_1, E_2,..E_i}\rangle)$ in a single step. Rather it will compute several lower bound-upper bound pairs on the search space and employ a state space search for each lower bound-upper bound pair until all the minimal solutions in that space are found. 


% , defined as follows:
%{\bf Definition}:	A \textit{Context Dependent Learning} task $ILP^{ctx}$ is a tuple $\langle B,M,D \rangle$, where $B$ is an Answer Set Program, called the background knowledge, $M$ defines the set of rules allowed in hypotheses (the hypothesis space) and $D$ is the dataset containing a series of context dependent examples $E_1,E_2,..., E_n$. Here each $E_i$ is a tuple $\langle O_i,E_{i}^{+}, E_{i}^{-} \rangle$ where,  $O_i$ is a logic program, called \textit{observation} , $E^{+}$ is a set of positive ground literals and $E^{-}$ is a set of negative ground literals. A hypothesis $H$ is an inductive solution of $T$ (written as $H \in ILP^{ctx}(B,M,D)$) \textit{iff}
%	$H\cup B \cup O_i \vdash E_{i}^{+}, ~\forall i=1...n$,  and   
%	$H\cup B \cup O_i \not \vdash E_{i}^{-}, ~\forall i=1...n$. 

Another learning aspect we will address is allowing previously undefined predicates in $H$. Currently, in all ILP systems and algorithms the mode definitions of rules in $H$ do not allow previously undefined predicates. 
% However, such predicates are needed to learn rules whose conditional part may contain universal quantifications. 
The universal quantification plays an important role in the specification of temporal operators ``{\em always\/}'' and ``{\em until\/}''.

\noindent\underline{Supporting Uncertainty}

During Phase II, we will add support for uncertainty. 
%In recent years several extensions of ASP have been proposed to account for uncertainty.  One approach, as used in P-log, enhances ASP with some explicit probability assignments and additional implicitly derived probability numbers. In another approach, as used in LP\^MLN (and even in the much earlier introduced ASP with weak constraints), weight assignments are given explicitly and probability numbers are computed on top of them.  
We propose to develop a new formalism that incorporates reasoning features crucial to assurance with respect to autonomous systems, such as causality, counter-factual reasoning, use of weights for computing probabilities, 
% instead of direct probability numbers, which are hard to estimate for assurance rules) from which probabilities are computed, 
and probabilistic non-monotonicity -- the ability to have new possible models as a result of new information. 
Probabilistic non-monotonicity is useful for non-\naive\ conditioning, as \naive\ conditioning may fail with new sensors that ask different kind of questions and as a result need a new set of possible worlds \cite{halpern2003, baral2007}. 
% (Such a situation is described in \cite{halpern2003} and addressed in the context of ASP by us in \cite{baral2007}.) 

%To allow non-\naive\ conditioning we need rules of the form:
%\begin{equation*}
%{\tt random}(o(Y) : \{X : p(X)\}) :- a_1, \ldots , a_n, \ {\bf not} \ b_1, \ldots, \ {\bf not} \ b_m. 
%\end{equation*}
%The above rules state that $o(Y)$ is a random variable where $Y$ can take the value from the set \{X : p(X)\} when $B$ is true. Such a construct allows non-monotonic specification of when $o(Y)$ is a random variable and what its extent is. Such non-monotonic specification of randomness is not possible in the popular uncertainty formalisms such as Markov Logic Networks or its recently proposed ASP variant $LP^{MLN}$. While P-log has such a construct, it lacks in its ability to specify arbitrary weights. 

\noindent\underline{Supporting Scaling}

During Phase III, we will focus on ASP solver speed improvements.  There exist excellent inference engines for basic ASP, but none of them has emphasized speed.  We will develop a more efficient system by modularizing the ASP programs and making domain-specific restrictions on the modules.  

Fortunately, we have many innovative ideas for improving the performance of such solvers by several orders of magnitude. Our ideas are derived from the state-of-the-art techniques developed by our teammate for solving Weighted Constraint Satisfaction Problems (WCSPs). 
% The success of most solvers for basic ASP is derived from that of SAT solvers. As a consequence, 
The WCSP serves as a substrate combinatorial problem for ASP augmented with probabilities and weights. In addition, the WCSP is representationally powerful enough to capture many other combinatorial problems in probabilistic reasoning. We have developed the state-of-the-art methods for solving WCSPs. We can apply these techniques to solve ASP augmented with probabilities and weights. Our intuitions include:
\begin{itemize}
\item The Constraint Composite Graph (CCG)~\cite{k08,k08a,k16} is a graph associated with a WCSP. It can be constructed in polynomial time.  Solving a WCSP is equivalent to solving the Minimum Weighted Vertex Cover (MWVC) problem on its CCG. 
% The CCG provides the first unifying mathematical framework for simultaneously exploiting the locality of interactions between the variables as well as the exact probabilities and weights in these interactions~\cite{}. Sufficient structure in locality of interactions translates to low tree width of the CCG; and sufficient structure in the probabilities and weights translates to bipartivity of the CCG.
\item The {\em half-integral\/} property of the MWVC problem can be exploited towards a kernelization procedure for the WCSP~\cite{cc08}. We can stage a polynomial-time maxflow algorithm that preprocesses a WCSP and fixes the optimal values of a large number of its variables. The remaining variables constitute the {\em core\/} of the combinatorial problem. 
% Our polynomial-time kernelization procedure is so powerful that on about $1/7^{th}$ of the WCSP benchmark instances, the core turns out to be empty, i.e., the problem instances are solved entirely in the polynomial-time phase of the algorithm without requiring search at all~\cite{}. In general, for the remaining core of the combinatorial problem, we have also developed the state-of-the-art MWVC solvers~\cite{}.
\item Message Passing (MP) is a well known technique used for large scale combinatorial problems~\cite{kf09}. It avoids exponential time complexity and constitutes an anytime algorithm.  It is known to work very well on several inference problems in Information Theory~\cite{yfw03}.
% However, its behavior was not well understood on Probabilistic Graphical Models or WCSPs until recently. In our recent work, 
We have shown that MP applied on the CCG of a WCSP produces solutions of much higher quality compared to MP applied directly on the WCSP~\cite{xu2017}.
\end{itemize}

% The recent ASP competition provides benchmark performance \cite{ASPC6}.  With a fixed problem structure in our DAC it is unlikely the multiple portfolio (algorithm selection) technique employed by ME-ASP would add any benefit for its implementation complexity, perhaps favoring the simpler WASP+DLV approach.  

ASP in general is an NP-complete problem. The stratified subset of ASP is quadratic. By having a modular language that separates the specification to a small non-stratified part and having a specialized solver for stratified ASP, we expect order of magnitude improvement in performance.  A recent work~\cite{cuteri} with similar ideas has shown good initial results. In addition, grounding can be done faster using a parallelized system in a multi-core environment.

Recent ASP competition results \cite{ASPC6} note the apparent performance benefit of domain modeling and specialization.  % They offer ``[t]hese advantages are owed to symmetry breaking, cutting down a vast number of redundant representations of solution candidates, as well as compact formulations, reducing the size of ground instantiations by some order of magnitude.'' 
Because we develop a paired DSL and solver implementation, such techniques are applicable to us for scalability enhancements.  The competition revealed that on decision problems like ours, the fastest general-purpose solvers produced solutions at a rate of $0.2$ problems per second on the reference Linux platforms.  We expect our specialized solvers to well exceed this performance level on our target application.

% the Multi-Engine ASP Solver (ME-ASP) \cite{MEASP} and WASP+DLV technologies performed the best in the competition's single-processor track.  ME-ASP average time on an array of grounded decision and query problems amounted to $0.2$ problems per second on the reference Linux platform.  However, ME-ASP uses algorithm selection to provide general-purpose coverage; our specialized solver should be expected to exceed this performance threshold on its target application.  

\paragraph{Integration and Demonstration}

We envision providing an API for content producers and consumers to use for interacting with the DAC.  Our design and verification tools and run-time monitors and operators will assert assurance argument structures and features by way of the API.  Any DAC control parameters will be done with a separate API.  We envision using a representational state transfer (REST) stateless specification and synthesizing software adapters for this API.  

As an example of high-level safety requirements, we would commence with explicit control system safety claims in our design activity.  A representative set is given in \cite{NPR7150} regarding completeness of requirements for safety-critical software.  These address conditions such as reachability of safe and unsafe states, operator overrides, error handling, and so on.  We will assert these as specific claims under a top-level safety claim for the assurance argument, then propagate the claims through our hierarchical design and hazard analysis procedure.  
% Various levels of abstraction or refinement of the models provide content for the argument, contributing various granularities of claim and evidence for composing the argument.   
% In some cases the design and verification steps will produce corresponding run-time monitors for conditional evidence gathering.  
Verification tools produce monitors to cover the behaviors of the system that could not otherwise be proven correct -- for which more information is needed.  We determine and justify the need for the run-time monitor and its conditional evidence as part of the design.  
Our DAC reasoning engine may be designed in as an integral part of the safety framework, thus having safety requirements of its own in the integrated system.  We will provide all integration support via software installation and API services, and support demonstration planning and conduct as requested. 




\begin{table}
\caption{Project Plan for TA3}
  \centering
  {\footnotesize
\begin{tabular}{|m{.6in}|m{5.55in}|} 
\hline
\textbf{Phase} & \textbf{Plan} 
\\\hline
Phase I & 
We will build the correspondence mechanisms for the safety-verification-monitor checks across the assurance case, focusing on contracts and constraints at various levels of abstraction in the design and implementation.  We update an existing DSL and ASP solver to focus on this LE-CPS application, and build an API for exchanging constructs and data with the DAC.  We show how the reasoning unfolds for successful and unsuccessful events.   
\\ \hline
Phase II & 
We will elaborate on our Phase I activities to include compositional arguments, uncertainty representations, learning-enabled ASP solvers, adjusting DSL, API, and solvers as required for successful demonstration.  We anticipate having more formal verification and monitor data available at different levels of granularity, both with impact to the DSL. Initial efforts for ASP solver speed scale-up will be undertaken.  
\\ \hline
Phase III & 
We will emphasize scale-up with a primary concern for speed. We will implement modular solvers tailored to our domain, runtime environment, and operating constraints. We  anticipate having more formal verification data available from TA1 at different levels of granularity, as well as more monitor types available from TA2, both with impact to the DSL.
\\ \hline
\end{tabular}
}
\label{tab:ta3:plan}
\end{table}

% During Phase I we will build the correspondence mechanisms for the safety-verification-monitor checks across the assurance case, and show how the case reasoning unfolds for both successful and unsuccessful events.   During Phase II we will expand these techniques to compositional arguments for multiple components acting together to comprise the system.   During Phase III we will further expand to include static analysis correspondence between source and binary images, and including results from dynamic analysis for other properties of interest to the assurance case.  


% note use of enumitem package and [leftmargin=*] to eliminate itemize indentation
\begin{table}
\caption{How TA3 Interfaces with TA1 and TA2}
  \centering
  {\footnotesize
\begin{tabular}{|m{.5in}|m{5.65in}|} 
\hline
\textbf{TA3 } & \textbf{Interfaces and Interaction} 
\\\hline
 TA1 & 
\begin{itemize}[leftmargin=*,topsep=3pt]
\setlength\itemsep{0em} 
%\setlength\parskip{0em}
%\setlength\parsep{0em}
\item Receive assurance-related content in the form of contracts, assumptions, and safety constraints.  Additional information such as  bounds for any learning conducted within the assurance case itself.  Much of this content becomes claims for the assurance case argument.   
\item Receive content in the form of property specifications, open and closed proof obligations, assumptions, interactions, and constraints that led to those results.  
\item Specify and implement an API for the assurance platform which accepts formal verification, simulation, and system testing claims along with assurance case. 
\end{itemize}
\\ \hline
TA2 & 
\begin{itemize}[leftmargin=*,topsep=3pt]
\setlength\itemsep{0em}
\item To support moving evidence into the DAC, TA3 will specify and implement an API for the assurance platform that can receive evidence, conditional evidence, and supporting data.  
\item Evaluate DACs and provide assurance measure when conditional evidence is received. 
\item To support visualization and logging, TA3 will provide an API and services for run-time monitoring of the DAC.
\item To support operational integrity, the design may require fault tolerance for DAC services.  TA3 will provide features for checkpoints, redundancy, roll-backs, or other integrity features. 
\end{itemize}
\\\cline{1-2}
\hline
\end{tabular}
}
\label{tab:ta3:if}
\end{table}




\begin{table}
\caption{Technical Challenges and Mitigation Approach for TA3}
  \centering 
{\footnotesize
\begin{tabular}{|m{1.3in}|m{4.85in}|} 
\hline
\textbf{Challenge} & \textbf{Mitigation Approach} 
\\\hline

Hazard analysis data volume  & 
The ASP technology we use for DAC can easily accommodate large volumes of data.  The ASP reasoning technique enables filtering the results into easily contrasted outcomes so that the practitioner can identify conflicts.\\ \hline

DAC design and implementation validity & 
We consider validation from four perspectives: {\em construct validity\/} (on the theory of DAC and its implementation), {\em conclusion validity\/} (claims to evidence), {\em  internal validity\/} (causal or logical relationships within an argument), and {\em external validity\/} (generalization of results).  
\\ \hline

Identifying sources of verification errors &
Potential sources of verification errors include: modeling errors; testing harnesses and stubbed behaviors; property specification errors; incomplete test results; interpretation (assumption) errors; soundness vs. completeness inference errors in formal methods; artifact misinterpretation or corruption; and tool configuration managements and operation errors. To mitigate these risks we impose assurance arguments on our tool chains.  
\\ \hline

ASP solver speed improvement by specialization to application &
Some expectation of availability and reliability of the DAC application will be imposed by the runtime environment. The challenges are accelerating the ASP solvers and handling time synchronization across multiple DAC instances.
% ; handling fresh and stale data mix; data service drops or corruption; handling defaults in the absence of data; recovery after stop; managing human intervention; and service degradations. 
We explore multiple forms of modular and potentially translated implementations.  
We incorporate the DAC into the system design, potentially including hard real-time and safety downmode features. \\ \hline
\end{tabular}
}
\label{tab:ta3:risk}
\end{table}



\clearpage
\newpage


\section{Management Plan}


The Principal Investigator for this effort is Dr. Craig Knoblock who is responsible for all aspects of the effort, will coordinate the parallel team efforts, and will ensure high levels of performance from individual team members.  The Co-P/I, Dr. Michael Orosz, will provide project management and will assist all performers in the execution of the project.    The project team is divided into three working groups (Figure~\ref{fig:org_chart}) corresponding to Technical Areas 1-3, however, members of each team contribute across all project activities.   Table~\ref{fig:Table_Mgmt} defines the major contributions of each project team member to the project tasks.

\begin{figure}[tbhp]
%\vspace{-25pt}
\includegraphics[width=6.0in]{./org-chart2.png}
\caption{\small Organization Chart}
\label{fig:org_chart}
\end{figure}

Coordination: To maximize collaboration and reduce risk to project failure from lack of communication and technical exchange, we plan to employ a wide variety of working styles and communication/coordination so that all can contribute.  At the core of our project will be regularly scheduled meetings bridging the diversely distributed team (Table~\ref{fig:Collaboration_Table}).  These meetings will address project status, identify challenges, implement risk mitigation strategies and participate in technology exchanges and system integration efforts (when appropriate)

\begin{table}[ht]
\caption{\small Project Meetings and Events}
  \centering
  {\footnotesize
\begin{tabular}{|m{3.15in}|m{3in}|} 
\hline
\textbf{Meeting} & \textbf{Frequency} 
\\\hline
Conference calls among investigators (discuss project status, address concerns and project risks) & Weekly
\\
\hline
Technical exchange and coordination meetings using Bluejeans or another videoconference technology & At least twice a month and more frequently as needed
  \\ 
\hline
Face-to-Face meetings (prior to P/I and demonstration meetings) & Every 3 to 6 months and more frequently (especially at the beginning of the project) as needed
 \\\cline{1-2}

\hline
\end{tabular}
}
\label{fig:Collaboration_Table}
\end{table}

\begin{table}[tbhp]
\caption{\small Key Project Team Member Responsibilities}
  \centering
  {\footnotesize
\begin{tabular}{| m{.75in} | m{3.9in}| m{1.5in}|} 
\hline
\textbf{Key Member} & \textbf{Responsibilities} & \textbf{Tasks} 
\\\hline
Dr.\ Craig Knoblock  & Principal Investigator responsible for project, leads TA 2 – Assurance Monitoring and Control.  Will lead the overall project and lead the TA2 team.  Served as the PI on many DARPA projects and has sucessfully led many large teams.    Effort on project:  25\% &
1.1.6, 1.2.2 1.2.3, 1.2.4, 1.3.4, 1.4.1, 
2.1.6, 2.2.2 2.2.3, 2.2.4, 2.3.4, 2.4.1, 
3.1.6, 3.2.2, 3.2.3, 3.2.4, 3.3.4, 3.4.1
\\
\hline
Dr.\ Michael Orosz & Co-Principal Investigator responsible managing the day-to-day operations of the project, assist technical teams as needed, coordinate with TA4 teams.    Has led many large complex multi-disciplined/multi-organizational projects in academic and industry environments.  Effort on project: 50\%
& 1.1.6, 2.1.6, 3.1.6, 1.4.1, 2.4.1, 3.4.1
  \\ 
\hline
Dr.\ Pierluigi Nuzzo 
& 
Co-Principal Investigator.  Leads the TA 1 - Design for Assurance team and conducts research on the formal methods for the design of the TA1 system.  Research experience on methodologies and tools for the design of cyber-physical systems; contracts, interfaces, and compositional methods for embedded system design; the application of automated formal methods and optimization theory to problems in embedded and cyber-physical systems.  Effort on project: 2 months/year (16.6\%)
& 
1.1.1, 2.1.1, 3.1.1 \\
\hline
Dr.\ Matthew Barry
& 
Key personnel.  Leads the TA 3 – Dynamic Assurance.   He will conduct the research on the dynamic assurance case language editors and parsers, the run-time system, and system integrations. Effort on project:  66\%
& 
1.3.2, 2.3.2, 3.3.2\\
\hline
Dr.\ Chitta Baral
& 
Key personnel responsible for learning assurance rules, supporting assurance rules with uncertainty and improving solver speed.  Expertise on ASP solvers, which will be used to reason about the assurance cases. Effort on project: 20\%
& 
1.3.1, 2.3.1, 3.3.1 \\
\hline
Dr.\ Doug Smith 
& 
Key personnel will support formal methods aspects of TA1, and lead the effort on abstract refinement. Expertise in field of automated correct-by-construction program generation.    Effort on project: 40\%
& 
1.1.5, 2.1.5, 3.1.5 \\
\hline
Dr.\ Henny Sipma
& 
Key personnel who will support the program verification tasks under TA1.  Will lead the effort on program verification.   Effort on project:  45\%
& 
1.1.5, 2.1.5, 3.1.5, 1.3.2, 2.3.2, 3.3.2 \\
\hline
Dr.\ Petros Ioannou
& 
Key personnel responsible providing and extending the assurance test bed, which will be available at the start of the project for autonomous vehicles.   Effort on project: 1 month/year (8.3\%)
& 
1.1.2, 2.1.2 (optional), 3.1.2 (optional)
\\
\hline
Dr.\ Satyandra Kumar Gupta
& 
Key Personnel providing autonomous command and control expertise to the TA-2 team.   Will lead the research on safety aware learning on TA2.   Past research on physics-aware decision making to facilitate automation.  Effort on project: 1 month/year (8.3\%)
& 
1.2.1, 2.2.1, 3.2.1 \\
\hline
Dr.\ Anoop Kumar 
& 
Key personnel providing support to the TA 2 project team.  Will lead the research on monitoring \& control and detecting distribution shifts.  Effort on project: 50\%
& 
1.2.1, 1.2.2, 1.2.3, 1.2.4, 2.2.1, 2.2.2, 2.2.3, 2.2.4, 3.2.1, 3.2.2, 3.2.3, 3.2.4\\
\hline
Dr.\ Satish Thittamaranahalli
& 
Key personnel developing scalable algorithms for TA1, TA2, and TA3 project teams.  Has extensive experience on scalable algorithm design, machine learning, and constraint reasoning.  Effort on project: 50\%
& 
1.2.1, 1.2.2, 1.2.3, 1.2.4, 2.2.1, 2.2.2, 2.2.3, 2.2.4, 3.2.1, 3.2.2, 3.2.3, 3.2.4, 1.1.4, 2.1.4, 3.1.4 \\
\hline
Dr.\ Ryan Goodfellow
& 
Key personnel providing support to the TA-1 project. Will lead the research on simulation-based testing.  Has extensive experience on simulation-based testing.  Effort on project:  30\%
& 
1.1.3, 2.1.3, 3.1.3 \\

\cline{1-2}

\hline
\end{tabular}
}
\label{fig:Table_Mgmt}
\end{table}



\newpage
\section{Personnel, Qualifications and Commitment}

{\bf Dr.\ Craig Knoblock}, the PI on this effort, is a Research Professor of both Computer Science and Spatial Sciences at the University of Southern California (USC) and Director of the Intelligent Systems Division at the USC Information Sciences Institute.   He received his Ph.D. from Carnegie Mellon University in computer science. 
%His research focuses on techniques for describing, acquiring, and exploiting the semantics of data.  
In previous projects he has worked on developing  scalable approaches to execution monitoring, accurate detection of sensor failures, and   automatic modeling and reconstruction of sensors.  He has published more than 300 journal articles, book chapters, and conference papers on these topics.  Dr. Knoblock is a Fellow of the Association for the Advancement of Artificial Intelligence (AAAI), a Distinguished Scientist of the Association of Computing Machinery (ACM), a Senior Member of IEEE, past President and Trustee of the International Joint Conference on Artificial Intelligence.
%and winner of the 2014 Robert S. Engelmore Award.  

{\bf Dr.\ Michael Orosz}, a Co-PI on this effort, is a Research Associate Professor of Civil and Environmental Engineering at the University of Southern California (USC) and Research Director of the Decision Systems Group at the USC Information Sciences Institute.  Dr. Orosz has over 30 years’ experience in commercial and government software development, basic and applied research, project management, academic research and has developed and deployed several commercially successful products.  His research interests are in machine learning and decision analytics as applied to intelligence analysis and autonomous command and control such as smart building controls.    Dr. Orosz has extensive experience in managing large complex multi-disciplined/multi-teamed research projects. %funded by DARPA, DHS, DoD, DoE, Industry, NASA, NRO, NSA and ONR.   
He received his Ph.D. in computer science from the University of California, Los Angeles.

{\bf Dr.\ Pierluigi Nuzzo}, a Co-PI on this project, is an Assistant Professor in the Department of Electrical Engineering at the University of Southern California. He received the Ph.D. in Electrical Engineering and Computer Sciences from the University of California at Berkeley. 
%in 2015, and the Laurea degree (MS) in electrical engineering (summa cum laude) from the University of Pisa, Italy, and the Sant'Anna School of Advanced Studies, Pisa, Italy.
%
%He has four years of research experience in analog and mixed signal circuit design as a researcher at IMEC, Leuven, Belgium, and over 10 years experience in design methodologies and tools for mixed-signal integrated circuits and cyber-physical systems, as a researcher at the University of Pisa, IMEC, UC Berkeley, and USC. 
His research interests
include: methodologies and tools for cyber-physical system and mixed-signal
system design; contracts, interfaces and compositional methods for embedded
system design; the application of formal methods and optimization theory to problems in embedded and cyber-physical systems and electronic design automation. 
%
Prof. Nuzzo received %First Place in the operational category and Best Overall
%Submission in the 2006 DAC/ISSCC Design Competition, 
a Marie Curie Fellowship
from the European Union in 2006, 
the University of California at Berkeley EECS
departmental fellowship in 2008, 
%the University of California at Berkeley Outstanding Graduate Student Instructor Award in 2013, 
the IBM Ph.D.
Fellowship in 2012 and 2014, 
%the Best Paper Award from the International Conference on Cyber-Physical Systems (ICCPS) in 2016, 
and the David J.~Sakrison Memorial Prize in 2016 for his doctoral research. 
%He is an author of 1 patent and over 60 publications.

{\bf Dr.\ Satyandra K. Gupta} is Smith International Professor in the Department of Aerospace and Mechanical Engineering at the University of Southern California. %Prior to joining the University of Southern California, he was a Professor in the Department of Mechanical Engineering and the Institute for Systems Research at the University of Maryland. He was the founding director of the Maryland Robotics Center and the Advanced Manufacturing Laboratory at the University of Maryland. 
He served as a program director for the National Robotics Initiative at the National Science Foundation from September 2012 to September 2014.  Dr. Gupta's interest is in the area of physics-aware decision making to facilitate automation. He has published more than 300 technical articles. He is a fellow of the American Society of Mechanical Engineers (ASME) and editor of ASME Journal of Computing and Information Science in Engineering. Dr. Gupta has received the Young Investigator Award from the Office of Naval Research in 2000, CAREER Award from the National Science Foundation in 2001, Presidential Early Career Award for Scientists and Engineers (PECASE) in 2001, Invention of the Year Award at the University of Maryland in 2007, Kos Ishii-Toshiba Award from ASME in 2011, and Excellence in Research Award from ASME in 2013.%, and Distinguished Alumnus Award from Indian Institute of Technology, Roorkee in 2014. %He has also received seven best paper awards at conferences.

{\bf Ryan Goodfellow} is a computer scientist at ISI working in combined cyber physical simulation and emulation platform development. His formal background is in simulation algorithms and modeling techniques using differential-algebraic equations (DAE). He has applied this knowledge in the CPS space by integrating DAE modeling languages and simulation engines with network testbeds to create comprehensive scientific experimentation platforms for cyber-physical systems. These experimentation platforms have been used in the power grid research space. %Ryan is a lead developer on the Deter network testbed, with a strong background in networked and distributed systems engineering. %He is also a combat veteran, serving as a non-commissioned officer and SIGINT team lead for a multi-functional intelligence team in Afghanistan.

{\bf Dr.\ Petros Ioannou} is a Professor in the Department of Electrical Engineering, Director of the Center for Advanced Transportation Technologies and Associate Director for Research for the DOT supported University Transportation Center at USC. He received his MS and PhD from the University of Illinois at Urbana Champaign in Mechanical and Electrical Engineering, respectively. His research interests are in robust adaptive control, vehicle dynamics and control, human factors and safety, automated vehicles, nonlinear systems and Intelligent transportation Systems.  He received the 2016 IEEE Transportation Technologies field award and the 2016 IEEE Control system society Transition to Practice Award. He is a Fellow of IEEE, IFAC and IET and author/coauthor of 8 books and over 400 papers.

{\bf Dr.\ Matthew Barry} will serve as lead for the TA3 tasks. %He will implement the dynamic assurance case language editors and parsers, the run-time system, and system integrations.  He will implement the assurance case arguments and the API for updating argument structure and content.  
Dr. Barry currently is CEO at Kestrel Technology LLC, and previously spent 20 years in NASA space mission operations at the Jet Propulsion Lab and Johnson Space Center.  At NASA Headquarters he led the introduction of dependability case requirements and plans for flight computing systems in upcoming manned space exploration missions, as well as the development of Agency-level software-related safety-critical control system requirements.  He recently served as a Principal Investigator on DHS/Cyber S\&T STAMP (Static Tool Analysis Modernization Program), DARPA CSFV (Crowd Sourced Formal Verification), three NASA Aeronautics R\&D projects, and the AFRL-sponsored Static Analysis of Numerical Algorithms project.  Dr. Barry earned BSME, MS, and PhD degrees in mechanical engineering, and an MBA degree, from Rice University.  

{\bf Dr.\ Henny Sipma} will support the program verification tasks under TA1.  %She is the key person behind the company's {\em KT Advance\/} and {\em KT Transferal\/} static analysis products, and the designer and programmer of the company's core {\em CodeHawk\/} abstract interpretation engine. 
Dr. Sipma currently is the CTO at Kestrel Technology LLC.  She has spent the past 10 years with Kestrel Technology as a static analysis expert; previously developed and taught static analysis techniques as senior research associate at Stanford University for eight years; and developed industrial process controls as an senior systems analyst at Shell.  She has been Principal Investigator or company lead on several recent R\&D projects for Federal agencies, including two projects under the IARPA STONESOUP (Securely Taking On New Executable Software of Uncertain Provenance) program; the DHS Cyber S\&T Gold Standard project; and the DARPA-sponsored STAC (Space-Time Analysis for Cybersecurity) and MUSE (Mining and Understanding Software Enclaves) programs.  Dr. Sipma earned 
%a BS degree in chemistry and an MS degree in chemical engineering at the University of Groningen in The Netherlands, and 
MS and PhD degrees in computer science from Stanford University.  

{\bf Dr.\ Douglas R.\ Smith} will support formal methods aspects of TA1, including the enforcement of safety properties and the generation of monitors.  He is President of Kestrel Technology LLC and Principal Scientist at Kestrel Institute.  He is a Fellow of the American Association of Artificial Intelligence (AAAI) and an ASE Fellow (Automated Software Engineering).  From 1986 to 2000, he taught an advanced graduate course on correct-by-construction software development at Stanford.  
%Dr. Smith has led the development of a series of software synthesis systems, including KIDS (Kestrel Interactive Development System), Specware, Designware, and Planware. 
%Applications domains have included a variety of complex high-performance planners and schedulers for the US Air Force.  He leads current projects on the generation of air mission plans and cyberoperations.  
Other recent projects focused on automated policy enforcement \cite{SmithD0703,SmithD08}, synthesis of secure network protocol codes, and the synthesis of high-performance constraint-solvers\cite{SmithD08c,SmithD13}.  Dr. Smith has over 30 years experience in the field of automated correct-by-construction program generation and has published over 100 papers. He has one patent.  He received the Ph.D. in Computer Science from Duke University% in 1979.  

{\bf Dr. Chitta Baral} is a Professor in the Department of Computer Science and Engineering at Arizona State University. He will support the TA3 efforts on Learning assurance rules, supporting assurance rules with uncertainty and improving solver speed. Dr. Baral has expertise in various aspects of autonomy and Artificial Intelligence. 
He wrote the first book on answer set programming (published by Cambridge University Press) the formal language behind our assurance rules. Some of his other works relevant to this proposal are: goal specification for autonomous systems, automatic construction of control rules for autonomous systems that satisfy given goals, combining machine learning with reasoning in various contexts, including image understanding. %He is the President of KR Inc. He is an associate editor of AIJ and has been an associate editor of JAIR.

{\bf Dr.\ Satish Kumar Thittamaranahalli (T. K. Satish Kumar)} leads the Collaboratory for Algorithmic Techniques and Artificial Intelligence (CATAI) at USC's Information Sciences Institute. He has published over 60 papers on numerous topics in Artificial Intelligence spanning such diverse areas as Constraint Reasoning, Planning and Scheduling, Probabilistic Reasoning, Robotics, Combinatorial Optimization, Approximation and Randomization, Heuristic Search, Model-Based Reasoning, Knowledge Representation and Spatio-Temporal Reasoning. %He %has served on the Program Committees of many international conferences in Artificial Intelligence
He and is a winner of the 2016 Best Robotics Paper Award and the 2005 Best Student Paper Award from the International Conference on Automated Planning and Scheduling. 
Dr. Kumar received his PhD in Computer Science from Stanford University. %In the past, he has also been a Visiting Student at the NASA Ames Research Center, a Postdoctoral Research Scholar at the University of California, Berkeley, a Research Scientist at the Institute for Human and Machine Cognition, a Visiting Assistant Professor at the University of West Florida, and a Senior Research and Development Scientist at Mission Critical Technologies.

\textbf{Dr.\ Anoop Kumar} is a senior computer scientist at USC ISI and has broad expertise in machine learning, statistical modeling, and software engineering.  Dr.\ Kumar is the technical lead on the DARPA RSPACE program and has played a vital role in developing a system that fuses air operations data from multiple sources, maintains world state, and issues warnings. Previously, he led the research and development of the BBN’s election forecasting system for the IARPA OSI program. %Dr.\ Kumar played a significant role in the DARPA DEFT program by developing a model to support integration of output from multiple NLP algorithms. He has contributed at the development to management levels on government research contracts and commercial projects. 
Dr.\ Kumar helped design and develop BBN's commercially available, hosted speech and medical transcription services offering. 

\begin{table}[!tbh]
\begin{footnotesize}
\vspace{-0.1in}

\begin{tabular}{lll}
\begin{tabular}[t]{|l|@{}c@{}|@{}c@{}|@{}c@{}|@{}c@{}|} \hline
Project & Status & \multicolumn{3}{ c| }{Hours} \\ \cline{3-5}
& & P1 & P2 & P3 \\ \hline



\multicolumn{5}{ |c| }{ \textbf{Craig Knoblock} } \\ \cline{1-5}
Safeguard & Pro & 770 & 641 & 641 \\ \cline{1-5}
ELICIT & Cur & 308 & 256 & 120 \\ \cline{1-5}
WTNIC & Cur & 11 & 0 & 0 \\ \cline{1-5}
EFFECT & Cur & 641 & 107 & 0 \\ \cline{1-5}
LinkedMaps & Cur & 203 & 25 & 0 \\ \cline{1-5}
PRINCESS & Cur & 608 & 96 & 0 \\ \cline{1-5}
SCHARP & Cur & 481 & 54 & 0 \\ \cline{1-5}
MINT & Pen & 650 & 534 & 285 \\ \cline{1-5}

\multicolumn{5}{ |c| }{ \textbf{Michael Orosz} } \\ \cline{1-5}
Safeguard & Pro & 1560 & 1300 & 1300  \\ \cline{1-5}
SMC/SY & Cur & 1803 & 0 & 0  \\ \cline{1-5}

\multicolumn{5}{ |c| }{ \textbf{Matthew Barry} } \\ \cline{1-5}
Safeguard & Pro & 2078 & 1690 & 1554 \\ \cline{1-5}
Starlite & Cur & 1840 & 1692 & 0 \\ \cline{1-5}



\multicolumn{5}{ |c| }{ \textbf{Anoop Kumar} } \\ \cline{1-5}
Safeguard & Pro & 1560 & 1300 & 1300 \\ \cline{1-5}

\end{tabular}
&
\begin{tabular}[t]{|l|@{}c@{}|@{}c@{}|@{}c@{}|@{}c@{}|} \hline
Project & Status & \multicolumn{3}{ c| }{Hours} \\ \cline{3-5}
& & P1 & P2 & P3 \\ \hline

\multicolumn{5}{ |c| }{ \textbf{Pierluigi Nuzzo} } \\ \cline{1-5}
Safeguard & Pro & 520 & 433 & 433  \\ \cline{1-5}
Mirage & Cur & 433 & 0 & 0  \\ \cline{1-5}

\multicolumn{5}{ |c| }{ \textbf{Satyandra Gupta} } \\ \cline{1-5}
Safeguard & Pro & 260 & 217 & 217 \\ \cline{1-5}
Human   & Cur & 22 & 0 & 0 \\ \cline{1-5}
Vehicles & Cur & 36 & 0 & 0 \\ \cline{1-5}
Robot & Cur & 116 & 0 & 0 \\ \cline{1-5}
Assembly & Cur & 33 & 0 & 0 \\ \cline{1-5}
Solar & Cur & 4 & 0 & 0 \\ \cline{1-5}

\multicolumn{5}{ |c| }{ \textbf{Petros Ioannou} } \\ \cline{1-5}
Safeguard & Pro & 260 & 217 & 217 \\ \cline{1-5}
CPS & Cur & 130 & 0 & 0 \\ \cline{1-5}

\multicolumn{5}{ |c| }{ \textbf{Ryan Goodfellow} } \\ \cline{1-5}
Safeguard & Pro & 936 & 780 & 780 \\ \cline{1-5}
STEAM & Cur & 416 & 0 & 0 \\ \cline{1-5}


\end{tabular}
&
\begin{tabular}[t]{|l|@{}c@{}|@{}c@{}|@{}c@{}|@{}c@{}|} \hline
Project & Status & \multicolumn{3}{ c| }{Hours} \\ \cline{3-5}
& & P1 & P2 & P3 \\ \hline

\multicolumn{5}{ |c| }{ \textbf{Chitta Baral} } \\ \cline{1-5}
Safeguard & Pro & 659 & 485 & 485 \\ \cline{1-5}
PostdocBP & Cur & 176 & 0 & 0 \\ \cline{1-5}
Languages & Pen & 528 & 264 & 264 \\ \cline{1-5}
CAREER & Pen & 88 & 44 & 44 \\ \cline{1-5}
CHS & Pen & 510 & 255 & 0 \\ \cline{1-5}

\multicolumn{5}{ |c| }{ \textbf{Doug Smith} } \\ \cline{1-5}
Safeguard & Pro & 1222 & 984 & 840 \\ \cline{1-5}
RSPACE & Cur & 342 & 0 & 0 \\ 
\cline{1-5}
PLANX & Cur & 154 & 0 & 0 \\ 
\cline{1-5}
HACCS & Pen & 923 & 769 & 769 \\ 
\cline{1-5}

\multicolumn{5}{ |c| }{ \textbf{Henny Sipma} } \\ \cline{1-5}
Safeguard & Pro & 1372 & 962 & 840 \\ \cline{1-5}
STAC & Cur & 797 & 0 & 0 \\ \cline{1-5}

\multicolumn{5}{ |c| }{ \textbf{Satish Thittamaranahalli} } \\ \cline{1-5}
Safeguard & Pro & 1560 & 1300 & 1300 \\ \cline{1-5}
MapF & Cur & 103 & 103 & 0 \\ \cline{1-5}

\end{tabular}
\end{tabular}

\end{footnotesize}
\caption{Individual commitments of key personnel}
\label{tab:Commitments}
\vspace{-0.2in}
\end{table}

\clearpage
\newpage
\section{Capabilities}


%\subsection{University of Southern California}
USC has strengths in number of areas that are closely related to the proposed work:
\begin{itemize}[itemsep=0pt,leftmargin=*]
\item Dr.\ Nuzzo 
%has over 10-year research experience in embedded system design, from mixed-signal chip design (analog-to-digital converters, frequency synthesizers, software-defined radio), to methodologies and tools for mixed-signal integrated circuits and Cyber-Physical Systems (CPSs), and the application of formal methods and optimization theory to problems in embedded and cyber-physical systems and electronic design automation.  
%His doctoral work 
has done extensive research on contracts and compositional methods for heterogeneous system design and design space exploration, with application to aircraft electric power systems and environmental control systems. His work has helped transition rigorous system design foundations, innovative design methodologies, and new systems engineering paradigms to industry (IBM, United Technologies). 
\item Dr.\ Satyandra K. Gupta has worked on autonomous surface vehicles, autonomous ground vehicles for operation on rugged terrains, and autonomous flapping wing aerial vehicles.   His group has developed a hierarchal decision making approach for realizing autonomous systems. 
%This approach combines task planning and assignment, deliberative trajectory planning, reactive collision avoidance behaviors, and trajectory tracking control layers. 
His group has also developed new methods for learning reactive behaviors in adversarial environments and COLREGS compliant trajectory planning. \item Dr.\ Knoblock has developed methods that learn the relationships between sensors to both identify failures and changes in sensor and reconstruct those sensors, providing estimates of the accuracy of the reconstructed sensors.  
\item Ryan Goodfellow has extensive experience in simulation based testing through high-fidelity CPS testbed environment development and operation, using the Deter network testbed as the core which has supported several large scale government projects from a variety of agencies and thousands of users. %we have developed sophisticated CPS experiments under programs such as NFS RIPS, NIST SmartCities and the DHS Cybersecurity showcase.
\item Dr.\ Ioannou %helped  design and implement adaptive cruise control systems in collaboration with Ford Motor Company, which was commercialized four years before any other company. He 
worked on several DOT funded projects on automated vehicles and intelligent highway systems where he demonstrated his vehicle control designs for safety and performance on actual automated vehicles in test trucks and I-15 highway.
\item Drs.\ Knoblock, Kumar, and Thittamaranahalli have developed highly scalable approaches for monitoring message traffic to identify potential problems and issue warnings and alerts. 
\item Dr. Thittamaranahalli has developed state-of-the-art methods for efficiently solving large-scale search and optimization problems. %These techniques will be applicable in TA2 for safety-aware learning and planning, in TA2 for assurance monitoring and control, and in TA3 for dynamic assessment of assurance cases.

\end{itemize}
%\subsection{Kestrel Technology LLC}

Kestrel Technology's strength is in program analysis, specifically static analysis of both source and binary targets.  The company performs applied R\&D and product development for a variety of static analysis applications  pivoting primarily on the abstract interpretation technique.  The company recently initiated development of program analysis applications using logical equivalence techniques. As a provider of verification evidence in the form of mathematical proofs, the company also has expertise in the design and development of assurance case arguments for high-integrity systems using such evidence. %The company is engaged in a partnership with Wind River Systems to develop program analysis tools for its embedded system developers.  Many of Wind River's customers must develop their products under safety and certification standards, including those using safety cases.  

   

%\subsection{Arizona State University}
Chitta Baral at Arizona State University has developed various software to learn assurance rules and various ASP solvers, which he has made available as open-source.

Most of the software carried forward for implementation or derivation is open source.  The single exception is Kestrel Technology's {\it KT Advance\/} static analysis tool (TA1), in particular the abstract interpretation engine therein, which is company proprietary and is US EAR export-controlled.   
%Owing to mixed funding for the development of that technology 
We will continue to provide the Federal government a restricted use license for that particular item.

There are no specialized facilities, data, or GFE required for this effort. 


\section{Statement of Work}
We propose work for TA 1 – TA 3 for all three phases. All tasks span the four years of the program. For each task we provide an objective, the high-level approach (focusing on the responsibilities of each contributing organization), and the specific approach and milestones planned for each task for each phase. On all tasks, we will deliver design documents, software implementations, demonstrations, and publications. With the exception of several tasks accomplished by Kesler Technology, LLC, all tasks that accomplished at a university (USC/ISI, USC, and ASU) are believed to be fundamental research.   
%\usepackage[table]{xcolor}

{\scriptsize

\begin{longtable} {|p{\textwidth} | }

\hline

\textcolor{blue} {\footnotesize {\textbf{Tasks 1.1.1, 2.1.1, 3.1.1 -Design for Assurance System Models and Formal Verification (USC)}}} \\ \hline
Objective:  Develop contract-based formalisms and mapping tools to represent and reason about LE-CPSs at multiple levels of abstraction and generate assurance cases.  Undertake scalable formal verification and synthesis via Satisfiability Modulo Convex Programming. \\ \hline
Approach:  Develop modeling formalisms to represent components and contracts for LE-CPSs, including physical plant (e.g., autonomous vehicle, sensors, actuators, environment, controllers, and learning components. Formalisms will encompass different control and learning architectures (e.g., neural networks, statistical methods, graphical models, ensemble methods, decision trees) and support mapping between abstractions.   Develop a formal domain-specific language to capture and formalize requirements on LE components, systems, and their dynamics as contracts.   Develop a unifying framework and efficient algorithms to reason about the combination of discrete and continuous dynamics and constraints in the presence of uncertainties in LE cyber-physical systems \\ \hline
Phase 1 (1.1.1):  Milestone 1: Develop initial design followed by development and testing of individual components.  Milestone 2:  Library of components and contracts for the autonomous vehicle application driver.  Milestone 4: Library of components and contracts for the platforms provided by TA4 performers. Extension of the methodology and to support up to 20 continuous dimensions and 2 learning components for the 2 application drivers from TA4.  Milestone 6: -Prototype toolkit (software package) for capturing requirements, for translating them into contracts, for analyzing and validating them using contract operations and relations.  Prototype toolkit for capturing probabilistic requirements and behaviors of LE components, systems, and their dynamics, for translating them into stochastic assume-guarantee contracts, for analyzing and validating them using contract operations and relations, and for synthesizing design and verification artifacts from contracts.  Extension of the SMC framework and toolkit to support reactive and robust task and trajectory planning in the presence of uncertainties. \\ \hline
Phase 2 (2.1.1) Milestone 7: Refinement of design.  Milestone 9: extension of methodology, design, toolkits and libraries to support 40 continuous dimensions, 4 LECs, 30\% monitoring overhead. Extension of the SMC framework and toolkit from Phase 1 to support verification and synthesis on system with 40 dimensions and 4 LECs.  Milestone 10: Demonstration of the SMC framework and toolkit.  Contribution to Phase II report and dissemination of the results in conferences and journals. \\ \hline
Phase 3 (3.1.1) Milestone 11: Update design based on Phase II demo.  Milestones 12-13:  extend methodology, design, toolkits and libraries to support 100 dimensions, 6 LECs and 10\% monitoring overhead.   Milestone 14: Undertake Phase III demonstration on both platforms and submit final project report. \\ \hline
\textcolor{blue} {\footnotesize {\textbf{Tasks 1.1.2, 2.1.2, 3.1.2: Design for Assurance Testbed (USC)} }}\\ \hline
Objective:  Develop a simulation test bed for data generation and LE algorithm testing, redesign and/or refinement.   Simulator used as the test bed until the TA4 platforms are available.   Test bed will be used for internal research/prototype after TA4 platform availability. \\ \hline
Approach:  Leverage previous work on microscopic traffic simulations in urban and rural environments using the commercial software VISSIM and Vortex Studio and built in extensions for automated driving.   Develop testbed for autonomous vehicles in road/off-road environments to allow LEs to collect data, learn and make control decisions on line and in real time by simulating scenarios. The testbed together with analytical tools used to refine and redesign LEs and control algorithms by taking into account effects revealed by the simulation and not accounted for in the design stage.    In the event the TA4 platforms are not available, the test bed will be extended further by integrating all the LE components, controllers and sensors for demonstration purposes and evaluation of the proposed methodology. \\ \hline
Phase 1 (1.1.2):  Milestones 1-2:  Extension of existing simulator test beds.  Milestones 3-5:  Testing of individual components under normal and unpredicatble situations and demonstrating the results in VISSIM under several different driving scenarios. \\ \hline
Phase 2 (2.1.2) – Optional:  Milestones 7-8:  Extension of existing simulator test beds to support the TA1-TA3 teams.  Milestones 9-10:  Support demonstration of technology capable of supporting 40 dimensions, 4 LECs and 30\% monitoring overhead. \\ \hline
Phase 3 (3.1.2) – Optional:  Milestones 11-12:  Extension of existing simulator test beds to support the TA1-TA3 teams.  Milestones 13-14:  Support demonstration of technology capable of supporting 100 dimensions, 6 LECs and 10\% monitoring overhead. \\ \hline
\textcolor{blue} {\footnotesize {\textbf{Tasks 1.1.3, 2.1.3, 3.1.3: Design for Assurance Simulation Based Testing (USC/ISI)}}} \\ \hline
Objective:  Develop external Discrete Control Mechanisms for OpenModelica.  Develop/package virtual-machine based static time dilation systems. Undertake network testbed integration and develop physical system behavioral analysis tooling. \\ \hline
Approach:  Leverage previous external discrete control mechanisms for DAEs, implement similar facilities for OpenModelica to allow LEs to observe and control a physical system over a network. Contributions pushed back upstream to OpenModelica project.  Implement DieCast for modern libvirt.  Develop tooling to deploy integrated CPS models on the Deter network testbed. Apply modern DAE control theory in the form Modelica analysis packages usable by non DAE experts. \\ \hline
Phase 1 (1.1.3):  Milestones 1-2:  Initial CPS simulation concept and components.  Milestones 3-5:  Testing of individual components under normal and unpredictable situations and demonstrating the results capable of meeting 20 dimensions, 2 LECs and 50\% or under monitoring overhead conditions.   Milestone 6: Demonstrate technology in Phase I demonstration, contribute to Phase I final report and disseminate software and publications. \\ \hline
Phase 2 (2.1.3):  Milestones 7-8:  Apply lessons learned from Phase I and extend existing simulations to support 30 dimensions, 3 LECs and 40\% monitoring overhead.  Milestones 9-10:  Support demonstration of technology capable of supporting 40 dimensions, 4 LECs and 30\% monitoring overhead.  Contribute to Phase II final report and disseminate software and publications. \\ \hline
Phase 3 (3.1.3):  Milestones 11-12:  Apply lessons learned from Phase II and extend existing simulations to support 70 dimensions, 5 LECs and 20\% monitoring overhead.  Milestones 13-14:  Support demonstration of technology capable of supporting 100 dimensions, 6 LECs and 10\% monitoring overhead.  Contribute to Phase III final report and disseminate software and publications. \\ \hline
\textcolor{blue} {\footnotesize {\textbf{Tasks 1.1.4, 2.1.4, 3.1.4: Scalable Algorithms for Formal Verification (USC/ISI)}}} \\ \hline
Objective: Develop innovative algorithms for scalable formal verification. \\ \hline
Approach: Use state-of-the-art techniques for solving combinatorial problems with discrete/continuous variables and hybrid constraints. \\ \hline
Phase 1 (Task 1.1.4): Milestones 1-2: Develop initial design plan and initial concepts. Milestones 3-5: Integrate framework that is capable of supporting 20 dimensions, 2 LECs and 0.1x trials to assurance. Milestone 6: Participate in Phase I demonstration, contribute to Phase I final report and disseminate software and publications. \\ \hline
Phase 2 (Task 2.1.4): Milestones 7-8: Apply lessons learned from Phase I and extend existing design to support 30 dimensions, 3 LECs and 0.05x trials to assurance. Milestones 9-10: Demonstrate technology capable of supporting 40 dimensions, 4 LECs and 0.01x trials to assurance. Participate in Phase II demonstration, contribute to Phase II final report and disseminate software and publications. \\ \hline
Phase 3 (Task 3.1.4): Milestones 11-12: Apply lessons learned from Phase II and extend design/approach to support 70 dimensions, 5 LECs and 0.005x trials to assurance. Milestones 13-14: Demonstrate technology capable of supporting 100 dimensions, 6 LECs and 0.001x trials to assurance. Complete integration of technology into TA4 platform. Contribute to Phase III final report and disseminate software and publications. \\ \hline
\textcolor{blue} {\footnotesize {\textbf{Tasks 1.1.5, 2.1.5, 3.1.5: Design for Assurance Program Verification (Kestrel Technology, LLC)}}} \\ \hline
Objective: Develop and integrate program analysis and monitor synthesis functionality with TA1 functions and services and integrate combined TA1 functions with TA4 platform. \\ \hline
Approach: Integrate existing analysis tools into development environment.  Design and implement abstract domains and properties for one or more modeling layers.  Design and implement analyzer front-end for modeling layers.  Implement test framework for verification tools.  Implement content providers and/or consumers for DAC via DAC API.  Leverage existing algorithms and tools to generate monitors for assumptions and unproven safety constraints. Integrate program analysis and monitor synthesis functionality with TA1 functions and services, integrate combined TA1 functions with TA4 platform.   Prepare software and data installation kits and operating instructions;install software and confirm configuration. \\ \hline
Phase 1 (1.1.5) : Milestones 1-2:  Initial framework design and unit tools, TA1-TA3 interfaces defined. Milestones 3-5:  Testing of individual components/tools capable of meeting 20 dimensions, 2 LECs and 50\% or under monitoring overhead conditions.   Milestone 6: Demonstrate technology in Phase I demonstration, contribute to Phase I final report and disseminate software and publications. \\ \hline
Phase 2 (2.1.5): Milestones 7-8:  Apply lessons learned from Phase I and extend existing design to support 30 dimensions, 3 LECs and 40\% monitoring overhead.  Milestones 9-10:  Support demonstration of technology capable of supporting 40 dimensions, 4 LECs and 30\% monitoring overhead.  Contribute to Phase II final report and disseminate software and publications. \\ \hline
Phase 3 (3.1.5): Milestones 11-12:  Apply lessons learned from Phase II and extend existing simulations to support 70 dimensions, 5 LECs and 20\% monitoring overhead.  Milestones 13-14:  Support demonstration of technology capable of supporting 100 dimensions, 6 LECs and 10\% monitoring overhead.  Contribute to Phase III final report and disseminate software and publications. \\ \hline
\textcolor{blue} {\footnotesize {\textbf{Tasks 1.1.6, 2.1.6, 3.1.6: System integration, deployment, and testing (USC/ISI)}}} \\ \hline
Objective: Develop and implement integration, testing and deployment plan supporting TA1 for all three phases. \\ \hline
Approach: Develop an internal TA1 integration and testing plan (unit tests, etc.) and, in close collaboration with TA2 and TA3 performers on project, develop an overall TA1-TA3 integration and testing plan.  Working with TA4 performers, extend and execute plan for TA4 platform (when available). \\ \hline
Phase 1 (1.1.6): Milestones 1-2:  Develop initial integration and testing plan and implement on unit testing.  Milestones 3-5:  Oversee integration and testing of TA1-TA3 components for system capable of supporting 20 dimensions, 2 LECs and 50\% or less monitoring overhead.   Milestone 6: Complete integration of technology into TA4 testbeds, contribute to Phase I final report and disseminate software and publications. \\ \hline
Phase 2 (2.1.6): Milestones 7-8:  Apply lessons learned from Phase I and extend existing integration and testing plan to support 30 dimensions, 3 LECs and 40\% monitoring overhead.  Milestones 9-10:  Support demonstration of technology capable of supporting 40 dimensions, 4 LECs and 30\% monitoring overhead.  Complete integration of technology into TA4 platforms.  Contribute to Phase II final report and disseminate software and publications. \\ \hline
Phase 3 (3.1.6): Milestones 11-12:  Apply lessons learned from Phase II and extend existing integration and testing plan to support 70 dimensions, 5 LECs and 20\% monitoring overhead.  Milestones 13-14:  Support demonstration of technology capable of supporting 100 dimensions, 6 LECs and 10\% monitoring overhead.  Complete integration of technology into TA4 platform.  Contribute to Phase III final report and disseminate software and publications. \\ \hline
\textcolor{blue} {\footnotesize {\textbf{Tasks 1.2.1, 2.2.1, 3.2.1: Safety Aware Learning (USC)} }}\\ \hline
Objective: Enable the system to learn efficiently without violating safety constraints. \\ \hline
Approach: Integrate LECs with search methods to select the optimal actions/maneuvers to maximize mission utility. \\ \hline
Phase 1 (Task 1.2.1): Milestones 1-2:  Develop initial design plan and initial concepts. Milestones 3-5:  Integrate two LECs with search methods and integrate into framework that is capable of supporting 20 dimensions, 2 LECs and 50\% or less monitoring overhead.   Milestone 6: Participate in Phase I demonstration, contribute to Phase I final report and disseminate software and publications. \\ \hline
Phase 2 (Task 2.2.1): Milestones 7-8:  Apply lessons learned from Phase I and extend existing design to support 30 dimensions, 3 LECs and 40\% monitoring overhead.  Milestones 9-10:  Support demonstration of technology capable of supporting 40 dimensions, 4 LECs and 30\% monitoring overhead.  Participate in Phase II demonstration.  Contribute to Phase II final report and disseminate software and publications. \\ \hline
Phase 3 (Task 3.2.1): Milestones 11-12:  Apply lessons learned from Phase II and extend design/approach to support 70 dimensions, 5 LECs and 20\% monitoring overhead.  Milestones 13-14:  Support demonstration of technology capable of supporting 100 dimensions, 6 LECs and 10\% monitoring overhead. Complete integration of technology into TA4 platform.  Contribute to Phase III final report and disseminate software and publications. \\ \hline
\textcolor{blue} {\footnotesize {\textbf{Tasks 1.2.2, 2.2.2, 3.2.2: Assurance Monitor and Guards (USC)}}} \\ \hline
Objective: Build scalable algorithms for assurance monitoring of architectural and safety constraints \\ \hline
Approach: Use physical models to reduce processing of sensor information for assurance monitoring. Use Variable Elimination to handle uncontrollable, Adversarially controlled, or unobservable variables \\ \hline
Phase 1 (Task 1.2.2): Milestones 1-2:  Develop initial design plan and initial concepts.  Milestones 3-5:  Develop monitors for two LECs and integrate into framework that is capable of supporting 20 dimensions, 2 LECs and 50\% or less monitoring overhead.  Develop APIs for integration with TA1 and TA3. Milestone 6: Participate in Phase I demonstration, contribute to Phase I final report and disseminate software and publications. \\ \hline
Phase 2 (Task 2.2.2): Milestones 7-8:  Apply lessons learned from Phase I, incorporate physical models of vehicle-environment interactions and extend existing design to support 30 dimensions, 3 LECs and incorporate physical models to bring down monitoring overhead to 40\% or less.   Milestones 9-10:  Support demonstration of technology capable of supporting 40 dimensions, 4 LECs and 30\% monitoring overhead.  Participate in Phase II demonstration.  Contribute to Phase II final report and disseminate software and publications. \\ \hline
Phase 3 (Task 3.2.2): Milestones 11-12:  Apply lessons learned from Phase II and identify core constraints to monitor and correlation between variables to support 70 dimensions, 5 LECs and 20\% monitoring overhead.  Milestones 13-14:  Support demonstration of technology capable of supporting 100 dimensions, 6 LECs and 10\% monitoring overhead.  Complete integration of technology into TA4 platform.  Contribute to Phase III final report and disseminate software and publications. \\ \hline
\textcolor{blue} {\footnotesize {\textbf{Tasks 1.2.3, 2.2.3, 3.2.3: System integration, deployment, and testing: (USC/ISI)}}} \\ \hline
Objective: Develop and implement integration, testing and deployment plan supporting TA2 for all three phases. \\ \hline
Approach: Develop an internal TA2 integration and testing plan (unit tests, etc.) and, in close collaboration with TA1 and TA3 performers on project, develop an overall TA1-TA3 integration and testing plan.  Working with TA4 performers, extend and execute plan for TA4 platform (when available). \\ \hline
Phase 1 (1.2.3): Milestones 1-2:  Develop initial integration and testing plan and implement on unit testing.  Milestones 3-5:  Oversee integration and testing of TA1-TA3 components for system capable of supporting 20 dimensions, 2 LECs and 50\% or less monitoring overhead.   Milestone 6: Complete integration of technology into TA4 testbeds, contribute to Phase II final report and disseminate software and publications. \\ \hline
Phase 2 (2.2.3): Milestones 7-8:  Apply lessons learned from Phase II and extend existing integration and testing plan to support 30 dimensions, 3 LECs and 40\% monitoring overhead.  Milestones 9-10:  Support demonstration of technology capable of supporting 40 dimensions, 4 LECs and 30\% monitoring overhead.  Complete integration of technology into TA4 platforms.  Contribute to Phase II final report and disseminate software and publications. \\ \hline
Phase 3 (3.2.3): Milestones 11-12:  Apply lessons learned from Phase II and extend existing integration and testing plan to support 70 dimensions, 5 LECs and 20\% monitoring overhead.  Milestones 13-14:  Support demonstration of technology capable of supporting 100 dimensions, 6 LECs and 10\% monitoring overhead.  Complete integration of technology into TA4 platform.  Contribute to Phase III final report and disseminate software and publications. \\ \hline
\textcolor{blue} {\footnotesize {\textbf{Tasks 1.2.4, 2.2.4, 3.2.4: Detecting Distributional Shifts (USC)}}} \\ \hline
Objective:  Develop a comprehensive framework to detect distribution shifts in LECs \\ \hline
Approach: Extend our prior work on sensor failure detection to distribution shifts.  Implement an approach that looks at single variable, sliding window, and distributions and employs classifiers and ensemble methods. \\ \hline
Phase 1 (Task 1.2.4): Milestones 1-2:  Develop initial design plan and initial concepts.  Milestones 3-5:   Develop framework that is capable of supporting 20 dimensions, 2 LECs and 50\% or less monitoring overhead. Extend sensor failure detection in BRASS effort to detect distributional shifts.  Milestone 6: Participate in Phase I demonstration, contribute to Phase I final report and disseminate software and publications. \\ \hline
Phase 2 (Task 2.2.1): Milestones 7-8:  Apply lessons learned from Phase I and  implement sliding window and sampling based methods to support 30 dimensions, 3 LECs and 40\% monitoring overhead.  Milestones 9-10:  Support demonstration of technology capable of supporting 40 dimensions, 4 LECs and 30\% monitoring overhead.  Participate in Phase II demonstration.  Contribute to Phase II final report and disseminate software and publications. \\ \hline
Phase 3 (Task 3.2.1): Milestones 11-12:  Apply lessons learned from Phase II and implement data reduction and machine learning techniques to support 70 dimensions, 5 LECs and 20\% monitoring overhead.  Milestones 13-14:  Support demonstration of technology capable of supporting 100 dimensions, 6 LECs and 10\% monitoring overhead.  Complete integration of technology into TA4 platform.  Contribute to Phase III final report and disseminate software and publications. \\ \hline
\textcolor{blue} {\footnotesize {\textbf{Tasks 1.3.1, 2.3.1, 3.3.1 - Checking Assurance Case Arguments for Dynamic Assurance – (ASU)}} }\\ \hline
Objective: Enhance assurance case DSL to accommodate learning of assurance rules.    Enhance Dynamic Assurance Case (DAC) implementation to support uncertainty.   Enable ASP solver speed improvements 
 \\ \hline
Approach: We will develop algorithms and an implemented module that can learn assurance rules from a set of input-output pairs. We will illustrate the scalability of our method as compared to existing Inductive Logic Programming methods.  We will develop a variant of L that incorporates various uncertainty and automated reasoning related features such as causality, counterfactual reasoning, use of weights for computing probabilities and probabilistic non-monotonicity.  We will develop a highly efficient ASP reasoning system (that forms the heart of our assurance case DSL) by modularizing the ASP programs and making domain specific restrictions (such as stratification on a big part of the program) on the modules \\ \hline
Phase 1 (Task 1.3.1): Milestones 1-2:  Develop initial design plan and initial concepts.  Milestones 3-5:  Integrate two LECs with search methods and integrate into framework that is capable of supporting 20 dimensions, 2 LECs and 50\% or less monitoring overhead.   Milestone 6: Participate in Phase I demonstration, contribute to Phase I final report and disseminate software and publications. \\ \hline
Phase 2 (Task 2.3.1): Milestones 7-8:  Apply lessons learned from Phase I and extend existing design to support 30 dimensions, 3 LECs and 40\% monitoring overhead.  Milestones 9-10:  Support demonstration of technology capable of supporting 40 dimensions, 4 LECs and 30\% monitoring overhead.  Participate in Phase II demonstration.  Contribute to Phase II final report and disseminate software and publications. \\ \hline
Phase 3 (Task 3.3.1): Milestones 11-12:  Apply lessons learned from Phase II and extend design/approach to support 70 dimensions, 5 LECs and 20\% monitoring overhead.  Milestones 13-14:  Support demonstration of technology capable of supporting 100 dimensions, 6 LECs and 10\% monitoring overhead.  Complete integration of technology into TA4 platform.  Contribute to Phase III final report and disseminate software and publications. \\ \hline
\textcolor{blue} {\footnotesize {\textbf{Tasks 1.3.2, 2.3.2, 3.3.2 - Program Verification and Run-Time Monitoring for Dynamic Assurance (Kestrel Technology, LLC)}}} \\ \hline
Objective: Develop the DAC language, the API for DAC interaction between TA1/TA2/TA3 and implement the technology in the three phases \\ \hline
Approach: Develop initial DAC language and APIs and extend based on testing against internal and TA4 provided scenarios. \\ \hline
Phase 1 (Task 1.3.2): Milestone 6: An initial DSL grammar specification; a DAC API Specification, a program client/server protocol and content specification for use interacting with the DAC; initial learning-enabled solver; and integrated DAC API-solver software for the demonstration platform \\ \hline
Phase 2 (Task 2.3.2): Milestone 7:  Updated design/plans based on Phase I lessons learned. Milestone 10: deliver a program client/server protocol and content specification for use interacting with the DAC; initial uncertainty-enabled solver; and integrated DAC API-solver software for the demonstration platform. \\ \hline
Phase 3 (Task 3.3.2): Milestones 11:  Apply lessons learned from Phase II and extend design/plan.  Milestone 14: Deliver a program client/server protocol and content specification for use interacting with the DAC; final and modularity-enabled solver; and integrated DAC API-solver software for the demonstration platform.  \\ \hline
\textcolor{blue} {\footnotesize {\textbf{Tasks 1.3.3, 2.3.3, 3.3.3: Scalable Algorithms for Checking Assurance Arguments (USC/ISI)}}} \\ \hline
Objective: Develop innovative algorithms for efficient dynamic assessment of assurance cases. \\ \hline
Approach: Use state-of-the-art techniques for solving Weighted CSPs to solve ASPs with weights and probabilities. \\ \hline
Phase 1 (Task 1.3.3): Milestones 1-2: Develop initial design plan and initial concepts. Milestones 3-5: Integrate framework that is capable of supporting 20 dimensions, 2 LECs and 10 conditional evidences. Milestone 6: Participate in Phase I demonstration, contribute to Phase I final report and disseminate software and publications. \\ \hline
Phase 2 (Task 2.3.3): Milestones 7-8: Apply lessons learned from Phase I and extend existing design to support 30 dimensions, 3 LECs and 50 conditional evidences. Milestones 9-10: Demonstrate technology capable of supporting 40 dimensions, 4 LECs and 100 conditional evidences. Participate in Phase II demonstration, contribute to Phase II final report and disseminate software and publications. \\ \hline
Phase 3 (Task 3.3.3): Milestones 11-12: Apply lessons learned from Phase II and extend design/approach to support 70 dimensions, 5 LECs and 500 conditional evidences. Milestones 13-14: Demonstrate technology capable of supporting 100 dimensions, 6 LECs and 1000 conditional evidences. Complete integration of technology into TA4 platform. Contribute to Phase III final report and disseminate software and publications. \\ \hline
\textcolor{blue} {\footnotesize {\textbf{Tasks 1.3.4, 2.3.4, 3.3.4 - System integration, deployment, and testing: (USC/ISI)}} }\\ \hline
Objective: Develop and implement integration, testing and deployment plan supporting TA3 for all three phases. \\ \hline
Approach: Develop an internal TA3 integration and testing plan (unit tests, etc.) and, in close collaboration with TA1 and TA2 performers on project, develop an overall TA1-TA3 integration and testing plan.  Working with TA4 performers, extend and execute plan for TA4 platform (when available). \\ \hline
Phase 1 (1.2.3): Milestones 1-2:  Develop initial integration and testing plan and implement on unit testing.  Milestones 3-5:  Oversee integration and testing of TA1-TA3 components for system capable of supporting 20 dimensions, 2 LECs and 50\% or less monitoring overhead.   Milestone 6: Complete integration of technology into TA4 testbeds, contribute to Phase II final report and disseminate software and publications. \\ \hline
Phase 2 (2.2.3): Milestones 7-8:  Apply lessons learned from Phase II and extend existing integration and testing plan to support 30 dimensions, 3 LECs and 40\% monitoring overhead.  Milestones 9-10:  Support demonstration of technology capable of supporting 40 dimensions, 4 LECs and 30\% monitoring overhead.  Complete integration of technology into TA4 platforms.  Contribute to Phase II final report and disseminate software and publications. \\ \hline
Phase 3 (3.2.3): Milestones 11-12:  Apply lessons learned from Phase II and extend existing integration and testing plan to support 70 dimensions, 5 LECs and 20\% monitoring overhead.  Milestones 13-14:  Support demonstration of technology capable of supporting 100 dimensions, 6 LECs and 10\% monitoring overhead.  Complete integration of technology into TA4 platform.  Contribute to Phase III final report and disseminate software and publications. \\ \hline
\textcolor{blue} {\footnotesize {\textbf{Tasks 1.4.1, 2.4.1, 3.4.1 – Project Management: (USC/ISI)}}} \\ \hline
Objective: Provide overall project management for Phase 1.  Assist in system design, integration and testing.  Interface with TA4 performers to ensure collaboration \\ \hline
Approach:  Establish weekly status meetings among team members, collaboration platform (e.g., Dropbox), provide technical assistance to integration efforts, resolve programmatic issues, develop monthly, quarterly and final reports.  Schedule and participate in technical exchange meetings, assist in developing component interfaces, establish test procedures, prototype testing.  Meet with TA4 performers to discuss test scenarios, platform integration and performance issues \\ \hline
Phase 1 (1.2.3): Milestones 1-2:  Establish meeting schedules and collaboration platforms. Assist teams in developing design and undertaking unit testing.  Milestones 3-5: Assist integration and testing of TA1-TA3 components for system capable of supporting 20 dimensions, 2 LECs and 50\% or less monitoring overhead.   Milestone 6: Assist integration of technology into TA4 testbeds, contribute to Phase II final report (C) and disseminate software and publications. \\ \hline
Phase 2 (2.2.3): Milestones 7-8:  Apply lessons learned from Phase II and extend existing integration and testing plan to support 30 dimensions, 3 LECs and 40\% monitoring overhead.  Milestones 9-10:  Support demonstration of technology capable of supporting 40 dimensions, 4 LECs and 30\% monitoring overhead.  Complete integration of technology into TA4 platforms.  Contribute to Phase II final report and disseminate software and publications. \\ \hline
Phase 3 (3.2.3): Milestones 11-12:  Apply lessons learned from Phase II and extend existing integration and testing plan to support 70 dimensions, 5 LECs and 20\% monitoring overhead.  Milestones 13-14:  Support demonstration of technology capable of supporting 100 dimensions, 6 LECs and 10\% monitoring overhead.  Complete integration of technology into TA4 platform.  Contribute to Phase III final report and disseminate software and publications. \\ \hline
 
\end{longtable}
}


% \textcolor{red}{
% Please review the following project schedule outline and either comment or send Craig/Mike comments.   The milestones reflect the need to scale up as the project moves forward.   As communicated below, we plan to have an initial working system by 6 months (the first P/I meeting).  
% }

% Phase I (18 Months):
% \begin{itemize}
% \item 1 Month – Initial Design completed (Milestone 1)
% \item 3 Months – Individual components developed and tested, TA1, TA2 and TA3 Interface Design completed (Milestone 2)
% \item 6 Months (P/I Mtg) – Initial working system for Design Time (i.e., TA1 – TA3 interaction) – includes one LEC (Milestone 3)  [NOTE:  at this time, TA4 teams will be providing scenarios for the demonstration]
% \item 12 Months (P/I Mtg) – Working system for both Design Time and Operation Time (i.e, TA1, TA2 and TA3 interactions), supports 10 dimensions and 1 LEC (Milestone 4)
% \item 17 Months – Working system that supports 20 dimensions and 2 LECs.   Integrate into both TA4 platforms (Milestone 5)
% \item 18 Months (P/I Mtg) – Phase I demonstration on both TA4 platforms (Milestone 6)
% \end {itemize}
% Phase II (15 Months):
% \begin{itemize}
% \item 19 Months – Design review based on Phase I demo (lessons learned)
% \item 25.5 Months (P/I Mtg) – Refined system to support 30 dimensions, 3 LECs, and 40 percent monitoring overhead (Milestone 7)
% \item 32 Months – Working system that supports 40 dimensions, 4 LECs and 30 percent monitoring overhead.  Integrate into both TA4 platforms (Milestone 8)
% \item 33 Months (P/I Mtg) – Phase II demonstration on both TA4 platforms (milestone 9)
% \end {itemize}
% Phase III (15 Months):
% \begin{itemize}
% \item 34 Months – Design review based on Phase II demo (lessons learned)
% \item 40.5 Months (P/I Mtg) – Refined system to support 70 dimensions, 5 LECs and 20 percent monitoring overhead (Milestone 10)
% \item 47 Months – Working system that supports 100 dimensions, 6 LECs and 10 percent monitoring overhead (Milestone 11)
% \item 48 Months (P/I Mtg) – Phase III demonstration on both TA4 platforms (Milestone 12)
% \end {itemize}

% \textcolor{red}{SEE SoW TABLE in GOOGLE DOCS.   Mike has sent invite to team.   
% }
% \vspace{10pt}

% \textcolor{red}{
% For each defined task, please provide the details listed below.  Please include references to the milestones above (e.g., when listing deliverables).   For sub-tasks, please list and describe them.  In addition, please list start/stop dates (in months) based on the outline above.  Mike  will be inserting these sub-tasks into the master schedule that will show up later in this document.
% }
% \textcolor{blue}{
% \begin{itemize}
% \item A general description of the objective.
% \item A detailed description of the approach to be taken to accomplish each defined task/subtask.
% \item Identification of the primary organization responsible for task execution (prime contractor, subcontractor(s), consultant(s)), by name.
% \item A measurable milestone, (e.g., a deliverable, demonstration, or other event/activity that marks task completion).
% \item A definition of all deliverables (e.g., data, reports, software) to be provided to the Government in support of the proposed tasks/subtasks.
% \item Identify any tasks/subtasks (by the prime or subcontractor) that will be accomplished at a university and believed to be fundamental research.
% \end{itemize}
% }
\clearpage
\newpage
\section{Schedule and Milestones}

The schedule is shown in Figure~\ref{fig:sandm} and the milestones are listed in Table~\ref{tab:milestones}.

\begin{table}[ht]
\centering
\caption{The project has the following fourteen (14) milestones}

{\scriptsize
\begin{tabular}{|m{.25in}|m{.25in}|m{4.0in}|m{1.65in}|} 
\hline
Mile-stones & Month & Description & Deliverables \\ \hline
1 & 2 & Initial Design completed.  Design includes finalized research plans, identification of internal TA milestones, initial interfaces between the three TAs, planned interface with the TA4 platforms. &  \\ \hline
2 & 3 & Individual components developed and tested.   TA1, TA2 and TA3 Interface design completed & Quarterly Report \\ \hline
3 & 6 & Initial working system for Design Time (i.e., TA1 – TA3 interaction).  Continued development of TA2.  Supports includes one LEC.   First P/I meeting.   Review TA4 scenarios. & Quarterly Report, slide presentation \\ \hline
4 & 12 & Working system for both Design Time and Operation Time (i.e., TA1, TA2 and TA3 interactions), supports 10 dimensions and one LEC.  Second P/I meeting.   Initial discussions with TA4 teams on interfaces & Quarterly Report, slide presentation \\ \hline
5 & 17 & Working system that supports 20 dimensions and 2 LECs with no more that 50\% monitoring overhead, 10 conditional evidence monitors and 0.1x reduced trails to assurance.   Start integration effort into both TA4 platforms & Working system (software) available for integration into TA4 platforms.  Monthly perfomance and financial reports \\ \hline
6 & 18 & Phase I demonstration on both TA4 platforms & Phase I report, quarterly reports \\ \hline
7 & 19 & Design review based on Phase I demo (lessons learned). &  \\ \hline
8 & 25.5 & Prototype system capable of supporting 30 dimensions, 3 LECs, with no more than 40\% monitoring overhead, 50 conditional evidence and 0.05x reduced trails to assurance.   Third P/I meeting & Quarterly report \\ \hline
9 & 32 & Working system that supports up to 40 dimensions, 4 LECs, with no more than 30\% monitoring overhead, 100 conditional evidence monitors and 0.01x reduced trails to assurance.  Begin Integration into both TA4 platforms & Working system (software) available for integration into TA4 platforms.  Monthly perfomance and financial reports \\ \hline
10 & 33 & Phase II demonstration on both TA4 platforms & Phase II report, quarterly reports \\ \hline
11 & 34 & Design review based on Phase II demo (lessons learned) &  \\ \hline
12 & 40.5 & Refined system to support 70 dimensions, 5 LECs, 500 conditional evidences and 20\% monitoring overhead – Forth P/I meeting & Quarterly report \\ \hline
13 & 47 & Working system that supports 100 dimensions, 6 LECs, 1000 conditional evidences, .001x reduction in assurance trials and 10\% monitoring overhead & Working system (software) available for integration into TA4 platforms.  Monthly perfomance and financial reports \\ \hline
14 & 48 & Phase III demonstration on both TA4 platforms. Phase III report, final project reporet. & Phase III report, quarterly reports, Final project report \\ \hline
\end{tabular}
}
\label{tab:milestones}
\end{table}

\begin{figure}[tbhp]
\begin{center}
\includegraphics[width=.95\textwidth]{figs/Safeguard_Schedule_V6}
\end{center}
\vspace{-.2in}\caption{Project schedule along with a summary of milestones.  The legend maps task color to organization primary responsible for the task. } 
\label{fig:sandm}
\end{figure}
 

% \section{Level of Effort by Task \textcolor{red}{[Mike/Lisa - 1 pages]}}

% \textcolor{blue}{
% \begin{itemize}
% \item Will be a separate spreadsheet
% \item
% \end{itemize}
% }

\section*{Appendix A: Team Members and Other Information}
\addcontentsline{toc}{section}{Appendix A: Team Members and Other Information}

\baades{This section is mandatory and must include all of the following
components. If a particular subsection is not applicable, state “NONE”.}

\vspace{1ex}

\noindent
\textbf{Team Member Identification}:

\vspace{1ex}

\baades{Provide a list of all team members including the
prime, subcontractor(s), and consultant(s), as applicable. Identify specifically
whether any are a non-US organization or individual, FFRDC and/or Government
entity.}

\begin{centering}

%\small
\begin{tabular}{|p{1.8in}|p{1in}|p{1.1in}|p{0.7in}|p{0.8in}|p{0.7in}|}
\hline
 \textbf{Name} &  \textbf{Role} & \textbf{Organization} & \textbf{Non-US Org?}  & \textbf{Non-US Ind?} &  \textbf{FFRDC or Gov} \\
 \hline
Craig A. Knoblock & Prime & USC & N & N & N\\ \hline
Michael Orosz & Prime & USC & N &  N & N \\ \hline
Satish Thittamaranahalli & Prime & USC & N &  Y & N \\ \hline
Ryan Goodfellow & Prime & USC & N &  N & N \\ \hline
Anoop Kumar & Prime & USC & N & N & N \\ \hline
Satyandra Gupta & Prime & USC & N & N & N \\ \hline
Pierluigi Nuzzo & Prime & USC & N & Y & N \\ \hline
Petros Ioannou & Prime & USC & N & N & N \\ \hline
Chitta Baral & Subcontractor & ASU & N & N & N \\ \hline
Matt Barry & Subcontractor & Kestrel Technology & N & N & N \\ \hline
Douglas Smith & Subcontractor & Kestrel Technology & N & N & N \\ \hline
Henny Sipma & Subcontractor & Kestrel Technology & N & Y & N \\ \hline
\end{tabular} 
\end{centering}

\vspace{1ex}

\noindent
\textbf{Government or FFRDC Team Member Proof of Eligibility to Propose}: NONE

\baades {If
none of the team member organizations (prime or subcontractor) are a
Government entity or FFRDC, state “NONE”.}

\vspace{1ex}

\noindent
\textbf{Government or FFRDC Team Member Statement of Unique Capability}: NONE

\vspace{1ex}

\noindent
\textbf{Organizational Conflict of Interest Affirmations and Disclosure}: NONE

\vspace{1ex}

\noindent
\textbf{Intellectual Property (IP)}: 
\baades {
If no IP restrictions are intended, state “NONE”.
The Government will assume unlimited rights to all IP not explicitly identified as
having less than unlimited rights in the proposal.
For all technical data or computer software that will be furnished to the
Government with other than unlimited rights, provide (per Section VI.B.1) a list
describing all proprietary claims to results, prototypes, deliverables or systems
supporting and/or necessary for the use of the research, results, prototypes
and/or deliverables. Provide documentation proving ownership or possession of
appropriate licensing rights to all patented inventions (or inventions for which a
patent application has been filed) to be used for the proposed project.
}
\begin{centering}

%\small
\begin{tabular}{|p{2.1in}|p{1.2in}|p{1.4in}|p{2in}|}
\hline
\multicolumn{4}{|c|}{COMMERCIAL ITEMS }\\ \hline 
 \textbf{Technical Data, Computer Software To be Furnished With Restrictions} &  \textbf{Basis for Assertion} & \textbf{Asserted Rights Category} & \textbf{Name of Person Asserting Restrictions}  \\  \hline
KT Advance & Developed with mixed funding. & Restricted & David Kulich, Contracts Manager, Kestrel technology, LLC.\\ \hline
\end{tabular} 
\end{centering}

\vspace{1ex}

\noindent
\textbf{Human Subjects Research (HSR)}: NONE

\vspace{1ex}

\noindent
\textbf{Animal Use}: NONE

\vspace{1ex}

\noindent
\textbf{Representations Regarding Unpaid Delinquent Tax Liability or a Felony
Conviction under Any Federal Law}: 
%NONE

\begin{enumerate}
\item
The proposer is  not a corporation that has any unpaid Federal tax liability that has been assessed, for which all judicial and administrative remedies have been exhausted or have lapsed, and that is not being paid in a timely manner pursuant to an agreement with the authority responsible for collecting the tax liability,
\item
The proposer is not a corporation that was convicted of a felony criminal violation under a Federal law within the preceding 24 months.
\end{enumerate}

\vspace{1ex}

\noindent
\textbf{Cost Accounting Standards (CAS) Notices and Certification}:
\baades{
For any proposer who submits a proposal which, if accepted, will result in a CAS-compliant
contract, must include a Disclosure Statement as required by 48 CFR
9903.202. The disclosure forms may be found at
http://www.whitehouse.gov/omb/procurement\_casb
If this section is not applicable, state “NONE”. For further information regarding
this subject, please see www.darpa.mil/work-with-us/additional-baa.
}
NONE


%\section{Appendix B \textcolor{red}{[No Page Count]}}

\section{References}
\bibliographystyle{acm} 
\bibliography{TA3/ta3,TA2/ta2,TA1/ta1}
\end{document}